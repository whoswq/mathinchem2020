\chapter{Bonmian 动力学}
    \section{20210104:Bohmian动力学}

        类比经典情况有守恒的连续性方程
        \begin{equation*}
            \frac {\partial \rho}{\partial t} + \bm{\nabla} \cdot (\rho \bm{v}) = 0
        \end{equation*}
        在量子力学中也有守恒量:
        \begin{equation*}
            \int |\psi(\bm{x},t)|^2 \mathrm{d}x = 1
        \end{equation*}
        因此上述连续性方程依然成立。

        考虑含时Schrodinger方程
        \begin{equation*}
            \mathrm{i}\hbar \frac {\partial}{\partial t} \psi(\bm{x},t) = \bigg(\frac {\hbar^2}{2m}\bm{\nabla + \bm{v}}\bigg)\psi
        \end{equation*}
        波函数可以写成
        \begin{equation*}
            \psi(\bm{x},t) = \sqrt{\rho} \mathrm{e}^{\frac {\mathrm{i}S(\bm{x},t)t}{\hbar}}
        \end{equation*}
        将波函数的形式代入Schrodinger方程可以得到上式以及Hamilton-Jacobian方程
        \begin{align*}
            \frac {(\frac {\partial S}{\partial x})^2}{2m} + V(\bm{x}) - \frac {\hbar^2}{2m} \frac {\nabla^2 \sqrt{\rho}}{\sqrt{\rho}} = -\frac {\partial S}{\partial t}
        \end{align*}
        前两项可以和经典情况类比,第三项是量子力学而来,定义
        \begin{equation*}
            Q(\bm{x},t) = - \frac {\hbar^2}{2m} \frac {\nabla^2 \sqrt{\rho}}{\sqrt{\rho}}
        \end{equation*}
        为\textbf{量子势}。速度场即为
        \begin{equation*}
            \bm{v} = \bm{M}^{-1}\bm{p} = \bm{M}^{-1}\frac {\partial S}{\partial \bm{x}}
        \end{equation*}
        从Hamilton-Jacobian方程出发,可以得到运动方程。如果是经典的Hamilton-Jacobian方程,没有量子势
        \begin{equation*}
            \frac {(\frac {\partial S}{\partial x})^2}{2m} + V(\bm{x}) = -\frac {\partial S}{\partial t}
        \end{equation*}
        考虑作用量的全导
        \begin{equation*}
            \frac {\mathrm{d}S}{\mathrm{d}t} = \frac {\partial S}{\partial t} + \frac {\partial S}{\partial \bm{x}_t} \dot{\bm{x}_t} = \frac 12 \bigg(\frac {\partial S}{\partial \bm{x}_t}\bigg)^{\mathrm{T}} \bm{M}^{-1} \frac {\partial S}{\partial \bm{x}_t} - V(\bm{x})
        \end{equation*}
        之后可以对位置求偏导,最终得到
        \begin{align*}
            \dot{\bm{x}}_t &= \bm{M}^{-1} \frac {\partial S(\bm{x}_t,t)}{\partial \bm{x}_t}\\
            \dot{\bm{p}}_t &= -\frac {\partial V(\bm{x}_t)}{\partial \bm{x}_t} - \frac {\partial Q(\bm{x}_t)}{\partial \bm{x}_t}
        \end{align*}
        这称为\textbf{量子轨线方程}。如果对$\rho$求全导,
        \begin{equation*}
            \frac {\mathrm{d}\rho}{\mathrm{d}t} = \frac {\partial \rho}{\partial t} + \frac {\partial \rho}{\partial \bm{x}_t} \dot{\bm{x}}_t = -\rho \bm{\nabla} \cdot \dot{\bm{x}}_t
        \end{equation*}

        波函数的求解可以转化为用0时刻的求解
        \begin{equation*}
            \psi(\bm{x},t) = \langle \bm{x} | \mathrm{e}^{-\frac {\mathrm{i}\hat{H}t}{\hbar}} | \psi(\bm{x},0) \rangle = \sum_n \langle \bm{x}|\phi_n \rangle \langle \phi_n |\psi_0 \rangle \mathrm{e}^{-\frac {\mathrm{i}E_n t}{\hbar}}
        \end{equation*}

        例如,
        \begin{equation*}
            \psi(x,0) =  \bigg(\frac {2a}{\pi}\bigg)^{\frac 14} \mathrm{e}^{-\alpha (x - x_{\mathrm{eq}})^2 +\frac {\mathrm{i}{\hbar}} p_{\mathrm{eq}}(x - x_\mathrm{eq})}
        \end{equation*}
        则
        \begin{align*}
            S(x_0,0) &= p_\mathrm{eq} (x_0 - x_\mathrm{eq})\\
            p_0 &= \frac {\partial S}{\partial x_0} = p_\mathrm{eq}
        \end{align*}

        根据经典情况统计物理的结果
        \begin{equation*}
            \langle A \rangle =\frac 1Z \int \rho(x,p)A(x,p) \mathrm{d}x\mathrm{d}p
        \end{equation*}
        其中 
        \begin{align*}
            \rho &= \frac 1Z \mathrm{e}^{-\beta H(x,p)}\\
            Z &= \int \frac 1{2\pi \hbar} \mathrm{e}^{-\beta H(x,p)} \mathrm{d}x \mathrm{d}p
        \end{align*}
        现在要问,量子力学有没有同样的形式?

        定义
        \begin{equation*}
            \mathrm{Tr} (\hat{A}\hat{B}) = \frac 1{2\pi\hbar}\int A(x,p)B(x,p) \mathrm{d}x \mathrm{d}p
        \end{equation*}
        重要的是求出两个算符在量子力学下的形式。
        \begin{align*}
            A_\mathrm{w}(x,p) &= \langle x - \frac {\Delta}2 | \hat{A} | x + \frac {\Delta}2 \rangle \mathrm{e}^{\frac {\mathrm{i}\Delta p}{\hbar}}\\
            B_\mathrm{w}(x,p) &= \langle x - \frac {\Delta}2 | \hat{B} | x + \frac {\Delta}2 \rangle \mathrm{e}^{\frac {\mathrm{i}\Delta p}{\hbar}}
        \end{align*}
        可以验证,如果$\hbar \to 0$, 应有$A_\mathrm{w} \to A_\mathrm{cl}$, $B_\mathrm{w} \to B_\mathrm{cl}$. 

        可以表示出概率密度分布
        \begin{equation*}
            \hat{\rho} = |\psi \rangle \langle \psi |
        \end{equation*}
        可以在位置空间和动量空间给出概率密度分布。并可以验证
        \begin{align*}
            \int \rho_\mathrm{w}(x,p) \mathrm{d}x &= \langle p|\hat{\rho}|p \rangle\\
            \int \rho_\mathrm{w}(x,p) \mathrm{d}p &= \langle x|\hat{\rho}|x \rangle
        \end{align*}
        要求物理量期望值可以写成
        \begin{align*}
            \langle \psi | \hat{B} | \psi \rangle &=  \langle \psi |\sum_n | \phi_n \rangle \langle \phi_n | \hat{B} | \psi \rangle\\
            &= \sum_n \langle \phi_n | \hat{B} | \psi \rangle \langle \psi  | \phi_n \rangle\\
            &= \mathrm{Tr} (\hat{B}|\psi \rangle \langle \psi |)
        \end{align*}
        它等价为
        \begin{equation*}
            \langle \psi|\hat{B}|\psi \rangle = \frac 1{2\pi \hbar} \int \rho_\mathrm{w}(x,p) B_\mathrm{w}(x,p) \mathrm{d}x\mathrm{d}p
        \end{equation*}
        \bibliographystyle{plain}
        \bibliography{ref_chp_9}    