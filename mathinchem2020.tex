\documentclass[12pt]{article}

\title{\heiti 化学中的数学}
\author{\kaishu 蒋然}
\usepackage{amsmath}
\usepackage{amssymb}
\usepackage{geometry}
\usepackage[version=3]{mhchem}
\geometry{a4paper, centering, scale=0.8}
\newtheorem{law}{定律}
\newtheorem{ded}{推论}
\newtheorem{thm}{定理}
\newtheorem{asg}{作业}
\usepackage{graphicx}
\usepackage[UTF8]{ctex}
\usepackage{fontspec}
\usepackage{setspace}
\usepackage{bm}

% 字体设置
\setmainfont{Times New Roman}
% 如果不是Mac系统请注释掉下面的两行
%\setsansfont{Helvetica}
%\setCJKsansfont{STHeitiSC-Medium}

\usepackage{makecell}
\newcommand{\addcell}[2][4]{\makecell{\zihao{#1}\textsf{#2}}}
\usepackage{titlesec}
\usepackage{booktabs}

% 设置图注、表注
\usepackage{caption}
\usepackage{bicaption}
\captionsetup{labelsep=quad, font={small, bf}, skip=2pt}
\renewcommand\figurename{图}
\renewcommand\tablename{表}
\DeclareCaptionOption{english}[]{
    \renewcommand\figurename{图}
    \renewcommand\tablename{表}
}
\captionsetup[bi-second]{english}
\begin{document}
\maketitle
\tableofcontents

\section{20200925:正则方程}

    经典力学中常用的独立变量为位置$x$和动量$p$, 且满足关系
    \begin{align*}
        \dot{x} = \frac pm\\
        \dot{p} = -\frac {\partial V}{\partial x}
    \end{align*}

    首先研究HCl分子。每个原子的坐标有3个自由度,总共是6个自由度。而这个分子总体有3个平动自由度,2个转动自由度,还剩余1个振动自由度。振动自由度的能量由\textbf{势能面}来描述。势能面是两个原子的距离$r$的函数,且
    \begin{equation*}
        \lim_{r \to \infty} V(r) = 0
    \end{equation*}
    当$r$减小时,势能逐渐减小,有一个极小值,对应的两原子距离称为平衡位置$r_\mathrm{eq}$, 然后再减小$r$时,势能增大,最后达到
    \begin{equation*}
        \lim_{r \to 0^+} V(r) = +\infty
    \end{equation*}

    实际上在平衡位置附近,我们把这个振动自由度近似为谐振子模型。通过改变势能零点的定义,我们总可以把势能写为
    \begin{equation*}
        V(r) = \frac 12 k(r-r_\mathrm{eq})^2
    \end{equation*}
    根据势能的形式可以写出力的形式
    \begin{equation*}
        F = -\frac {\partial V}{\partial r} = -k(r-r_\mathrm{eq})
    \end{equation*}
    做变换$x = r - r_\mathrm{eq}$, 可以将势能写为
    \begin{equation*}
        V(x) = \frac 12 kx^2
    \end{equation*}
    也可以将位置和动量对时间导数写为
    \begin{align*}
        \dot{x} = \frac pm\\
        \dot{p} = -kx
    \end{align*}
    现在求解这个运动方程:
    \begin{align*}
        \ddot{x} = \frac {\dot{p}}m = -\frac {kx}{m}
    \end{align*}
    这是一个二阶常微分方程,求解得到通解
    \begin{align*}
        x &= A \cos{\omega t} + B \sin{\omega t}\\
        p &= -{Am\omega} \sin{\omega t} + {Bm \omega} \cos{\omega t}
    \end{align*}
    其中$\omega = \sqrt{\frac km}$. 如果给定初始条件
    \begin{align*}
        x(0) = x_0\\
        p(0) = p_0
    \end{align*}
    将这两个方程代入到通解中,得到
    \begin{align*}
        x &= x_0 \cos{\omega t} + \frac {p_0}{m\omega} \sin{\omega t}\\
        p &= p_0 \cos{\omega t} - {m\omega x_0} \sin{\omega t}
    \end{align*}

    体系的Hamilton函数为
    \begin{align*}
        H(x,p,t) &= \frac {p^2}{2m} + V(x)
    \end{align*}
    现在希望验算
    \begin{align*}
        H(x,p,t) = H(x,p,0),~~~~~\forall t
    \end{align*}
    为了证明这个成立,首先可以推导\textbf{正则方程}:
    \begin{align*}
        \frac {\partial H}{\partial x} &= \frac {\partial V}{\partial x} = -\dot{p}\\
        \frac {\partial H}{\partial p} &= \frac pm = \dot{x}
    \end{align*}
    因此
    \begin{align*}
        \frac {\mathrm{d}H}{\mathrm{d}t} &= \frac {\partial H}{\partial x} \dot{x} + \frac {\partial H}{\partial p} \dot{p} + \frac {\partial H}{\partial t} = \frac {\partial H}{\partial t}
    \end{align*}
    这个结论对任意正则方程成立的体系都成立。在谐振子模型中,Hamilton函数不显含时间,故
    \begin{equation*}
        \frac {\mathrm{d}H}{\mathrm{d}t} = 0
    \end{equation*}
    这个体系可以在\textbf{相空间}中描述,即把它的状态画在一个$(x,p)$的二维空间中,观察它随时间的变化。显然地谐振子体系在相空间中的轨迹应该是一个椭圆。
    \begin{align*}
        \frac {p^2}{2m} + \frac 12 kx^2 = E_0
    \end{align*}
    其周期为
    \begin{equation*}
        T = \frac {2\pi}{\omega}
    \end{equation*}
    事实上,对于任意的满足能量守恒的体系,在相空间中都可以得到一条封闭的曲线。
    \begin{asg}
        第1次作业第1题:一维四次势的周期轨道
    \end{asg}

    现在考虑质量是$x,p$的函数$m_\mathrm{eff}(x,p)$, 在这种情况下Hamilton函数为
    \begin{equation*}
        H(x,p) = \frac {p^2}{2m_\mathrm{eff}(x,p)} + V(x)
    \end{equation*}
    在这种情况下的运动方程为
    \begin{align*}
        \dot{x} &= \frac {\partial H}{\partial p} = \frac {p}{2m_{\mathrm{eff}}} - \frac {p^2}{2m_{\mathrm{eff}^2}} \frac {\partial m_\mathrm{eff}}{\partial p} \\
        \dot{p} &= -\frac {\partial H}{\partial x} = \frac {p^2}{2m_\mathrm{eff}^2} \frac {\partial m_\mathrm{eff}}{\partial x} + \frac {\partial V}{\partial x}
    \end{align*}
    这种情况下能量仍然守恒,因为Hamilton函数不显含时间,且正则方程成立。
    \begin{asg}
        第1次作业第2题:竖立粉笔的问题
    \end{asg}

\section{20200928:相空间不同时刻体积元的关系}
    匀变速直线运动,应当有
    \begin{align*}
        x(t) &= x(0) + vt + \frac 12 at^2\\
        &= x(0) + \dot{x}t + \frac 12 \ddot{x}t^2
    \end{align*}
    这相当于位置对时间作了Taylor展开,展开到二阶。但是为什么只考虑前两阶,而不考虑后面的项呢?

    可以这样考虑:在给定了Hamilton函数的情形下,正则方程最多只涉及到对时间的二阶导数,最终解出位置对时间的函数,以及动量对时间的函数只有两个待定常数,因此只用位置和动量初始的条件。

    现在开始研究一个多维体系,它的位置和动量分别不是一个标量,而是一个向量$\bm{x}, \bm{p}$. 如果系统的在$t$时刻的状态$(\bm{x}_t,\bm{p}_t)$对应一个相空间中的\textbf{体积元}:$\mathrm{d}\bm{x}_t\mathrm{d}\bm{p}_t$。如果给定初始条件$(\bm{x}_0,\bm{p}_0)$, 希望在正则方程成立的条件下,能够确定0时刻和$t$时刻的相空间体积元的关系。这个问题可以等效地理解为,将初始条件产生一个很小的偏差$(\mathrm{d}\bm{x}_0,\mathrm{d}\bm{p}_0)$,要求在$t$时刻的偏差和初始条件的关系。

    这实际上给出了两种研究问题的办法:一种是参考系不动,一种是参考系随着时间跟踪系统在相空间中的轨线进行运动。

    对于任意个不显含时间的函数$f(\bm{x}_t,\bm{p}_t)$, 它和$f(\bm{x}_0,\bm{p}_0)$的关系为:
    \begin{align*}
        \int f(\bm{x}_t,\bm{p}_t) \mathrm{d}\bm{x}_t\mathrm{d}\bm{p}_t = \int f(\bm{x}_0,\bm{p}_0)\bigg|\frac {\partial (\bm{x}_t, \bm{p}_t)}{\partial (\bm{x}_0,\bm{p}_0)}\bigg| \mathrm{d}\bm{x}_0 \mathrm{d}\bm{p}_0
    \end{align*}
    由此可知,算出Jacobi行列式的值是非常重要的。Jacobi行列式的对应矩阵写为
    \begin{align*}
        \begin{pmatrix}
            \frac {\partial \bm{x}_t}{\partial \bm{x}_0} & \frac {\partial \bm{x}_t}{\partial \bm{p}_0}\\
            \frac {\partial \bm{p}_t}{\partial \bm{x}_0} & \frac {\partial \bm{p}_t}{\partial \bm{p}_0}
        \end{pmatrix}
    \end{align*}

    我们可以把$t$时刻的状态写成初始条件的函数:
    \begin{align*}
        \bm{x}_t &= \bm{x}_t(\bm{x}_0, \bm{p}_0)\\
        \bm{p}_t &= \bm{p}_t(\bm{x}_0, \bm{p}_0)
    \end{align*}
    如果初始状态偏离$(\mathrm{d}\bm{x}_0,\mathrm{d}\bm{p}_0)$,那么
    \begin{align*}
        \bm{x}_t(\bm{x}_0+\mathrm{d}\bm{x}_0, \bm{p}_0+\mathrm{d}\bm{p}_0) &= \bm{x}_t(\bm{x}_0, \bm{p}_0) + \frac {\partial \bm{x}_t}{\partial \bm{x_0}} \mathrm{d}x_0 + \frac {\partial \bm{x}_t}{\partial \bm{p}_0} \mathrm{d}p_0\\
        \bm{p}_t(\bm{x}_0+\mathrm{d}\bm{x}_0, \bm{p}_0+\mathrm{d}\bm{p}_0) &= \bm{p}_t(\bm{x}_0, \bm{p}_0) + \frac {\partial \bm{p}_t}{\partial \bm{x_0}} \mathrm{d}x_0 + \frac {\partial \bm{p}_t}{\partial \bm{p}_0} \mathrm{d}p_0
    \end{align*}
    此处只考虑Taylor展开到一阶的结果。或者写成
    \begin{align*}
        \mathrm{d}\bm{x}_t &= \frac {\partial \bm{x}_t}{\partial \bm{x_0}} \mathrm{d}x_0 + \frac {\partial \bm{x}_t}{\partial \bm{p}_0} \mathrm{d}p_0\\
        \mathrm{d}\bm{p}_t &= \frac {\partial \bm{p}_t}{\partial \bm{x_0}} \mathrm{d}x_0 + \frac {\partial \bm{p}_t}{\partial \bm{p}_0} \mathrm{d}p_0
    \end{align*}
    矩阵没有办法直接求出来,我们尝试对时间求导。
    \begin{align*}
        \frac {\mathrm{d}}{\mathrm{d}t} \bigg(\frac {\partial \bm{x}_t}{\partial \bm{x}_0}\bigg)_{\bm{p}_0} &= \bigg(\frac {\partial}{\partial \bm{x}_0} \frac {\mathrm{d}}{\mathrm{d}t} \bm{x}_t\bigg)_{\bm{p}_0} = \bigg(\frac {\partial}{\partial \bm{x}_0} \bigg(\frac {\partial H}{\partial \bm{p}_t}\bigg)_{\bm{x}_t}\bigg)_{\bm{p}_0} = \bigg(\frac {\partial^2 H}{\partial \bm{x}_t \partial \bm{p}_t}\bigg) \bigg(\frac {\partial \bm{x}_t}{\partial \bm{x}_0}\bigg)_{\bm{p}_0}+ \bigg(\frac {\partial^2H}{\partial \bm{p}_t^2}\bigg)_{\bm{x}_t} \bigg(\frac {\partial \bm{p}_t}{\partial \bm{x}_0}\bigg)_{\bm{p}_0}\\
        \frac {\mathrm{d}}{\mathrm{d}t} \bigg(\frac {\partial \bm{x}_t}{\partial \bm{p}_0}\bigg)_{\bm{x}_0} &= \bigg(\frac {\partial}{\partial \bm{p}_0} \frac {\mathrm{d}}{\mathrm{d}t} \bm{x}_t\bigg)_{\bm{x}_0} = \bigg(\frac {\partial}{\partial \bm{p}_0} \bigg(\frac {\partial H}{\partial \bm{p}_t}\bigg)_{\bm{x}_t}\bigg)_{\bm{x}_0} = \bigg(\frac {\partial^2 H}{\partial \bm{x}_t \partial \bm{p}_t}\bigg) \bigg(\frac {\partial \bm{x}_t}{\partial \bm{p}_0}\bigg)_{\bm{x}_0} + \bigg(\frac {\partial^2H}{\partial \bm{p}_t^2}\bigg)_{\bm{x}_t} \bigg(\frac {\partial \bm{p}_t}{\partial \bm{p}_0}\bigg)_{\bm{x}_0}\\
        \frac {\mathrm{d}}{\mathrm{d}t} \bigg(\frac {\partial \bm{p}_t}{\partial \bm{x}_0}\bigg)_{\bm{p}_0} &= \bigg(\frac {\partial}{\partial \bm{x}_0} \frac {\mathrm{d}}{\mathrm{d}t} \bm{p}_t\bigg)_{\bm{p}_0} = -\bigg(\frac {\partial}{\partial \bm{x}_0} \bigg(\frac {\partial H}{\partial \bm{x}_t}\bigg)_{\bm{p}_t}\bigg)_{\bm{p}_0} = -\bigg(\frac {\partial^2 H}{\partial \bm{x}_t^2}\bigg)_{\bm{p}t} \bigg(\frac {\partial \bm{x}_t}{\partial \bm{x}_0}\bigg)_{\bm{p}_0}- \bigg(\frac {\partial^2H}{\partial \bm{p}_t \partial \bm{x}_t}\bigg) \bigg(\frac {\partial \bm{p}_t}{\partial \bm{x}_0}\bigg)_{\bm{p}_0}\\
        \frac {\mathrm{d}}{\mathrm{d}t} \bigg(\frac {\partial \bm{p}_t}{\partial \bm{p}_0}\bigg)_{\bm{x}_0} &= \bigg(\frac {\partial}{\partial \bm{p}_0} \frac {\mathrm{d}}{\mathrm{d}t} \bm{p}_t\bigg)_{\bm{x}_0} = -\bigg(\frac {\partial}{\partial \bm{p}_0} \bigg(\frac {\partial H}{\partial \bm{x}_t}\bigg)_{\bm{p}_t}\bigg)_{\bm{x}_0} = -\bigg(\frac {\partial^2 H}{\partial \bm{x}_t^2}\bigg)_{\bm{p}t} \bigg(\frac {\partial \bm{x}_t}{\partial \bm{p}_0}\bigg)_{\bm{x}_0}- \bigg(\frac {\partial^2H}{\partial \bm{p}_t \partial \bm{x}_t}\bigg) \bigg(\frac {\partial \bm{p}_t}{\partial \bm{p}_0}\bigg)_{\bm{x}_0}
    \end{align*}
    由此可以得到
    \begin{align*}
        \frac {\mathrm{d}}{\mathrm{d}t}
        \begin{pmatrix}
            \frac {\partial \bm{x}_t}{\partial \bm{x}_0} & \frac {\partial \bm{x}_t}{\partial \bm{p}_0}\\
            \frac {\partial \bm{p}_t}{\partial \bm{x}_0} & \frac {\partial \bm{p}_t}{\partial \bm{p}_0}
        \end{pmatrix}
        =
        \begin{pmatrix}
            \frac {\partial^2 H}{\partial \bm{x}_t \partial \bm{p}_t} & (\frac {\partial^2 H}{\partial \bm{p}_t^2})_{\bm{x}_t}\\
            -(\frac {\partial^2 H}{\partial \bm{x}_t^2})_{\bm{p}_t} & - \frac {\partial^2 H}{\partial \bm{x}_t \partial \bm{p}_t}
        \end{pmatrix}
        \begin{pmatrix}
            \frac {\partial \bm{x}_t}{\partial \bm{x}_0} & \frac {\partial \bm{x}_t}{\partial \bm{p}_0}\\
            \frac {\partial \bm{p}_t}{\partial \bm{x}_0} & \frac {\partial \bm{p}_t}{\partial \bm{p}_0}
        \end{pmatrix}
    \end{align*}
    将此处的Jacobi矩阵称为\textbf{稳定性矩阵},其含义是如果系统初始时刻状态变化很小,那么$t$时刻的变化也很小。
    \begin{asg}
        第1次作业第3题:Liouville定理的证明
    \end{asg}

\section{20201009:Liouville定理}

    设矩阵
    \begin{equation*}
        \bm{A} = 
        \begin{pmatrix}
            a_{11} & \cdots & a_{1n}\\
            \vdots & & \vdots\\
            a_{n1} & \cdots & a_{nn}\\
            \end{pmatrix}
    \end{equation*}
    它的行列式为
    \begin{equation*}
        \det{\bm{A}} = \sum_{j=1}^n (-1)^{i+j} a_{ij} \bm{A}_{ij}^*, \ \forall \ i
    \end{equation*}
    其中,$\bm{A}_{ij}^*$表示$a_{ij}$的代数余子式。定义$\bm{A}$的\textbf{伴随矩阵}$\bar{\bm{A}}$为
    \begin{equation*}
        \bar{\bm{A}}_{ij} = \bm{A}_{ji}^* 
    \end{equation*}
    矩阵的逆矩阵为
    \begin{equation*}
        \bm{A}^{-1} = \frac 1{\det{\bm{A}}} \bm{\bar{A}}
    \end{equation*}
    对行列式的求导并不是对每个元素求导再求行列式,而是依照下列方法:
    \begin{align*}
        \frac {\mathrm{d}}{\mathrm{d}t} \det{\bm{A}} = \sum_i \det{\tilde{\bm{A}_i}}
    \end{align*}
    其中,$\bm{A}_i$是只对第$i$行的所有元素对时间求导,其他元素不变得到的矩阵。进一步得到
    \begin{align*}
        \frac {\mathrm{d}}{\mathrm{d}t} \det{\bm{A}} &= \sum_i \det{\tilde{\bm{A}_i}} = \sum_{i=1}^n \sum_{j=1}^n \frac {\mathrm{d}a_{ij}}{\mathrm{d}t} \bm{A}_{ij}^*\\
        &= \mathrm{Tr} \bigg(\frac {\mathrm{d}\bm{A}}{\mathrm{d}t} \bar{\bm{A}}\bigg) = \mathrm{Tr} \bigg(\frac {\mathrm{d}\bm{A}}{\mathrm{d}t} \bm{A}^{-1}\bigg) \det{\bm{A}}
    \end{align*}
    将两边同时除以$\bm{A}$的行列式,得到
    \begin{equation*}
        \frac {\mathrm{d}}{\mathrm{d}t} \ln{\det{\bm{A}}} = \mathrm{Tr} \bigg(\frac {\mathrm{d}\bm{A}}{\mathrm{d}t} \bm{A}^{-1}\bigg)
    \end{equation*}
    如果
    \begin{equation*}
        \frac {\mathrm{d}}{\mathrm{d}t} \bm{A} = \bm{MA}
    \end{equation*}
    就有
    \begin{equation*}
        \frac {\mathrm{d}}{\mathrm{d}t} \ln{\det{\bm{A}}} = \mathrm{Tr}\ \bm{M}
    \end{equation*}
    对于上一节讲的Jacobi矩阵,有
    \begin{equation*}
        \bm{M} = 
        \begin{pmatrix}
        \frac {\partial^2 H}{\partial \vec{x}_t \partial \vec{p}_t} & (\frac {\partial^2 H}{\partial \vec{p}_t^2})_{\vec{x}_t}\\
        -(\frac {\partial^2 H}{\partial \vec{x}_t^2})_{\vec{p}_t} & - \frac {\partial^2 H}{\partial \vec{x}_t \partial \vec{p}_t}
        \end{pmatrix}
    \end{equation*}
    显然这个矩阵的迹为0,所以
    \begin{equation*}
        \frac {\mathrm{d}}{\mathrm{d}t} \det \bigg|\frac {\partial (\bm{x}_t,\bm{p}_t)}{\partial (\bm{x}_0,\bm{p}_0)} \bigg| = 0
    \end{equation*}
    但初始时刻显然Jacobi行列式为1,所以Jacobi行列式一直为1,就有
    \begin{equation*}
        \mathrm{d}\bm{x}_t \mathrm{d}\bm{p}_t = \mathrm{d}\bm{x}_0 \mathrm{d}\bm{p}_0
    \end{equation*}

    这个结论称为\textbf{Liouville定理}。注意到这个结论的推导只用到了正则方程,只要正则方程成立,这个结论就成立。

    如果定义一个概率密度$\rho(\bm{x},\bm{p})$,它满足归一化条件,且处处不小于0. 假设初始条件下在$\bm{x}_0,\bm{p}_0$位置有个体积元$\mathrm{d}\bm{x}_0 \mathrm{d}\bm{p}_0$,跟踪这个体积元经历的轨线,达到$\mathrm{d}\bm{x}_t \mathrm{d}\bm{p}_t$时,在这个体积元的概率应为不变的。这可以理解为,根据Liouville定理,最开始在体积元里面的状态仍然会在初始状态演化后的体积元里。这可以表述为
    \begin{equation*}
        \rho(\bm{x}_t,\bm{p}_t) = \rho(\bm{x}_0,\bm{p}_0)
    \end{equation*}
    它对任意的$t$都成立,则
    \begin{equation*}
        \frac {\mathrm{d}\rho}{\mathrm{d}t} = \frac {\partial \rho}{\partial t} + \frac {\partial \rho}{\partial \bm{x}_t}\dot{\bm{x}}_t  + \frac {\partial \rho}{\partial \bm{p}_t} \dot{\bm{p}}_t = 0
    \end{equation*}
    再利用正则方程,得到
    \begin{equation*}
        - \frac {\partial \rho}{\partial t} = \bigg(\frac {\partial \rho}{\partial \bm{x}_t}\bigg)^\mathrm{T} \frac {\partial H}{\partial \bm{p}_t} - \bigg(\frac {\partial \rho}{\partial \bm{p}_t}\bigg)^\mathrm{T} \frac {\partial H}{\partial \bm{x}_t}
    \end{equation*}
    定义\textbf{Poisson括号}为
    \begin{equation*}
        \{ \rho, H\} = \bigg(\frac {\partial \rho}{\partial \bm{x}_t}\bigg)^\mathrm{T} \frac {\partial H}{\partial \bm{p}_t} - \bigg(\frac {\partial \rho}{\partial \bm{p}_t}\bigg)^\mathrm{T} \frac {\partial H}{\partial \bm{x}_t}
    \end{equation*}
    则有
    \begin{equation*}
        - \frac {\partial \rho}{\partial t} = \{ \rho, H\}
    \end{equation*}
    这也是Liouville定理的一种形式。如果Hamilton函数满足形式
    \begin{equation*}
        H(\bm{x}_t,\bm{p}_t) = \frac 12 \bm{p}_t^\mathrm{T} \bm{M}^{-1} \bm{p}_t + V(\bm{x}_t)
    \end{equation*}
    则有
    \begin{equation*}
        - \frac {\partial \rho}{\partial t} = \bigg(\frac {\partial \rho}{\partial \bm{x}_t}\bigg)^\mathrm{T} \bm{M}^{-1} \bm{p}_t  - \bigg(\frac {\partial \rho}{\partial \bm{p}_t}\bigg)^\mathrm{T} \frac {\partial V}{\partial \bm{x}_t}
    \end{equation*}

    一种常见的分布:\textbf{Boltzmann分布}:
    \begin{equation*}
        \rho(\bm{x},\bm{p}) \propto \mathrm{e}^{-\beta H(\bm{x},\bm{p})}
    \end{equation*}

    如果一个分布满足
    \begin{equation*}
        \frac {\partial \rho}{\partial t} = 0
    \end{equation*}
    则称为\textbf{稳态分布}。但是即使不是稳态分布,它也会满足对时间的全导数是0。这也是Liouville定理的一个形式。
    \begin{asg}
        第2次作业第1题: Boltzmann分布是否为稳态分布?
    \end{asg}

    研究一个概率密度的时候,有两种方式:一种是研究密度对时间的偏导,看静止空间的概率密度的变化,这称为\textbf{Euler图象}。另一种方式是研究密度对时间的劝导,跟踪状态运动的轨线,研究这个密度体积元在不同的时间的位置,这称为\textbf{Lagrange图象}。

\section{20201012:Euler图象演化概率密度}

    Liouville定理有两种表述形式:
    \begin{equation*}
        -\frac {\partial \rho}{\partial t} = \{ \rho,H \}
    \end{equation*}
    以及 
    \begin{equation*}
        \frac {\mathrm{d}\rho}{\mathrm{d}t} = 0
    \end{equation*}
    第一种形式下,$\rho = \rho(x, p ,t)$, 第二种形式下$\rho = \rho(x_t,p_t,t)$. 分别表示了Euler和Lagrange两种图象。

    回顾描述HCl分子的振动的例子,我们可以用Morse势来描述这个振动:
    \begin{equation*}
        V(x) = D_e (1- \mathrm{e}^{-a(r-r_\mathrm{eq})})^2 = D_e(1-\mathrm{e}^{-ax})^2
    \end{equation*}
    其中有$a>0$, 在平衡位置附近可以使用谐振子近似。写出其Boltzmann分布
    \begin{equation*}
        \rho(x,p,0) = \frac 1Z \mathrm{e}^{-\beta (\frac {p^2}{2m} + \frac 12 m\omega^2 x^2)} 
    \end{equation*}
    由概率密度的归一化,可以得到配分函数的值,这里涉及到Gauss函数的积分
    \begin{align*}
        I &= \int_0^{+\infty} \mathrm{e}^{-ax^2} x^{n} \mathrm{d}x
    \end{align*}
    令$t = ax^2$, 则$\mathrm{d}t = 2ax\mathrm{d}x$
    所以
    \begin{align*}
    I &= \int_0^{+\infty} \mathrm{e}^{-t} \bigg(\frac ta\bigg)^{\frac n2} \frac {\mathrm{d}t}{\sqrt{at}}\\
    &= \frac 1{2a^{\frac {n+1}2}} \int_0^{+\infty} \mathrm{e}^{-t} t^{\frac {n-1}2} \mathrm{d}t\\
    &= \frac {\Gamma(\frac {n+1}2)}{2a^{\frac {n+1}2}}
    \end{align*}
    据此算出配分函数
    \begin{equation*}
        Z = \int \mathrm{e}^{-\beta (\frac {p^2}{2m} + \frac 12 m\omega^2 x^2)} \mathrm{d}x\mathrm{d}p = \frac {2\pi}{\beta \omega}
    \end{equation*}
    从量纲上分析,在配分函数中少了$\mathrm{d}x\mathrm{d}p$的量纲。本质上应该除以$2\pi\hbar$, 相当于对相空间做了量子化。于是
    \begin{equation*}
        Z = \frac 1{\beta \hbar \omega}
    \end{equation*}
    就是无量纲的配分函数。

    回到用Morse势描述HCl的振动的问题,Morse势的常数$a$可以用谐振子近似的$\omega$进行估计。令$x \to 0 $,对$V(x)$在平衡位置附近作Taylor展开,展开到二阶。
    \begin{equation*}
        V(x) = D_e a^2 x^2 + o(x^2)
    \end{equation*}
    它与谐振子近似一致,因此
    \begin{equation*}
        \frac 12 m\omega^2x^2 = D_e a^2 x^2
    \end{equation*}
    于是
    \begin{equation*}
        \omega = \sqrt{\frac {2D_ea^2}m}
    \end{equation*}
    \begin{asg}
        第2次作业第2题:构造\ce{H2}分子的Morse势
    \end{asg}
    \begin{asg}
        第2次作业第3题:以Boltzmann分布为初始分布,在Morse势,Euler图象下演化\ce{H2}的$t$时刻的分布。
    \end{asg}
    事实上,对双原子分子HCl, 它有6个自由度,3个平动,2个转动,所以我们可以只用振动自由度来描述HCl的分子结构。

\section{20201016:Lagrange图象演化概率密度}

    除了用Euler图象来演化密度以外,也可以用Lagrange图象来演化密度。由
    \begin{equation*}
        \frac {\mathrm{d}}{\mathrm{d}t} \rho(x_t,p_t,t) = 0
    \end{equation*}
    可以得到$t$时刻的概率密度为
    \begin{equation*}
        \rho(x,p,t) = \int \rho(x_0,p_0,0)\delta(x-x_t(x_0,p_0)) \delta(p-p_t(x_0,p_0)) \mathrm{d}x_0\mathrm{d}p_0
    \end{equation*}
    这里引入了$\delta$函数。$\delta$函数满足
    \begin{align*}
        \delta(x-x_0) &= 0, \ \forall \ x \neq x_0\\
        \int_{-\infty}^{+\infty} \delta(x-x_0) \mathrm{d}x &= 1\\
        \int_{-\infty}^{+\infty} f(x)\delta(x-x_0) \mathrm{d}x &= f(x_0)
    \end{align*}
    现在希望给$\delta$函数给一个形式,让它和上面满足的性质自洽:
    可以利用Fourier变换及其逆变换的定义
    \begin{align*}
        \frac 1{\sqrt{2\pi}} \int_{-\infty}^{+\infty} f(x)\mathrm{e}^{\mathrm{i}kx}\mathrm{d}x &= F(k)\\
        \frac 1{\sqrt{2\pi}} \int_{-\infty}^{+\infty} F(k)\mathrm{e}^{-\mathrm{i}kx}\mathrm{d}k &= f(x)
    \end{align*}
    于是有
    \begin{align*}
        f(x_0) &= \frac 1{\sqrt{2\pi}} \int_{-\infty}^{+\infty} \frac 1{\sqrt{2\pi}} \int_{-\infty}^{+\infty} f(x)\mathrm{e}^{\mathrm{i}kx}\mathrm{d}x \mathrm{e}^{-\mathrm{i}kx_0}\mathrm{d}k\\
        &= \frac 1{2\pi} \iint f(x)\mathrm{e}^{\mathrm{i}k(x-x_0)}\mathrm{d}x\mathrm{d}k\\
        &= \frac 1{2\pi} \iint f(x)\mathrm{e}^{\mathrm{i}k(x-x_0)}\mathrm{d}k\mathrm{d}x
    \end{align*}
    于是可以写出$\delta$函数为
    \begin{equation*}
        \delta(x-x_0) = \frac 1{2\pi} \int_{-\infty}^{+\infty} \mathrm{e}^{\mathrm{i}k(x-x_0)}\mathrm{d}k
    \end{equation*}

    某个物理量的期望定义为
    \begin{equation*}
        \langle B(t) \rangle = \int \rho(x,p,t) B(x,p) \mathrm{d}x\mathrm{d}p
    \end{equation*}
    回到用Morse势描述HCl的振动的问题,在这个问题下,初始时刻为Boltzmann分布时,
    \begin{align*}
        \langle x \rangle &= 0\\
        \langle x^2 \rangle &= \frac 1{\beta m \omega^2}\\
        \Delta x &= \sqrt{\langle x^2 \rangle - \langle x \rangle ^2} = \frac 1{\sqrt{\beta m \omega^2}}
    \end{align*}

    \begin{asg}
        第3次作业第2题:以Boltzmann分布为初始分布,在Morse势,Lagrange图象下演化\ce{H2}的$t$时刻的分布。
    \end{asg}

\section{20201019:多自由度振动的频率计算}

    利用上一节得到的$\delta$函数,可以计算在$t$时刻的物理量期望为
    \begin{align*}
        \langle B(t) \rangle &= \int \rho(\bm{x},\bm{p},t) B(\bm{x},\bm{p}) \mathrm{d}\bm{x}\mathrm{d}\bm{p}\\
        &= \int \int \rho(\bm{x}_0,\bm{p}_0,0)\delta(\bm{x}-\bm{x}_t(\bm{x}_0,\bm{p}_0)) \delta(\bm{p}-\bm{p}_t(\bm{x}_0,\bm{p}_0)) \mathrm{d}\bm{x}_0\mathrm{d}\bm{p}_0 B(\bm{x},\bm{p}) \mathrm{d}\bm{x}\mathrm{d}\bm{p}\\
        &= \int \int \delta(\bm{x}-\bm{x}_t(\bm{x}_0,\bm{p}_0)) \delta(\bm{p}-\bm{p}_t(\bm{x}_0,\bm{p}_0)) B(\bm{x},\bm{p}) \mathrm{d}\bm{x}\mathrm{d}\bm{p} \rho(\bm{x}_0,\bm{p}_0,0) \mathrm{d}\bm{x}_0\mathrm{d}\bm{p}_0\\
        &= \int B(\bm{x}_t,\bm{p}_t) \rho(\bm{x}_0,\bm{p}_0,0)\mathrm{d}\bm{x}_0\mathrm{d}\bm{p}_0
    \end{align*}
    这意味着,只用初始概率密度也可以得到$t$时刻的物理量的期望。

    现在研究复杂一些的\ce{H2O}分子的振动。它总共有3个原子,所以9个自由度。平动3个自由度,转动也有3个自由度,因此振动是3个自由度。3个振动自由度分别为剪切振动、对称伸缩振动和不对称伸缩振动。
    \begin{asg}
        第3次作业第3题:水分子的简谐振动分析
    \end{asg}
    水分子的O-H振动波数约为3700 cm$^{-1}$, 剪切振动波数约为1600 cm$^{-1}$, 伸缩振动1个周期应当约为20.8 fs, 剪切振动波数约为9 fs. 而1 a.u. = 0.024 fs.即可据此估计模拟过程中的时间步长。

    对于水分子,定义其坐标为
    \begin{equation*}
        \bm{x} = 
        \begin{pmatrix}
            \bm{x}_\mathrm{O}\\
            \bm{x}_{\mathrm{H1}}\\
            \bm{x}_{\mathrm{H2}}
        \end{pmatrix}
    \end{equation*}
    并给定了其势能$V(\bm{x})$, 给出质量矩阵
    \begin{equation*}
        \bm{M} = \mathrm{diag} \{m_1,...,m_9 \} = 
        \begin{pmatrix}
            m_1 & \cdots & 0\\
            \vdots & \ddots & \vdots\\
            0 & \cdots & m_9
        \end{pmatrix}
    \end{equation*}
    其中,$m_1,m_2,m_3$等于氧原子的质量,$m_4,...,m_9$等于氢原子的质量。

    定义\textbf{Hessian}矩阵$\bm{\mathcal{H}}$为 
    \begin{equation*}
        \bm{\mathcal{H}}_{ij} = \frac 1{\sqrt{m_i}} \frac {\partial^2 V}{\partial x_i \partial x_j} \frac 1{\sqrt{m_j}}
    \end{equation*}
    它的单位为s$^{-2}$. 显然地,这是一个实对称矩阵,可以由正交矩阵作对角化:
    \begin{equation*}
        \bm{T}^\mathrm{T} \bm{\mathcal{H}T = \Omega}
    \end{equation*}
    其中$\bm{T}$为\textbf{正交矩阵},满足 
    \begin{equation*}
        \bm{T}^\mathrm{T}\bm{T} = \bm{TT}^\mathrm{T} = \bm{I}
    \end{equation*}
    并且
    \begin{equation*}
        \bm{\Omega} = \mathrm{diag} \{\omega_1^2, ..., \omega_9^2 \}
    \end{equation*}
    这样就得到了角频率,$\omega_j$对应的能量为$\hbar \omega_j$。总共得到了9个模式的频率,其中3个模式的频率对应平动,3个模式频率对应转动(平动转动的频率趋于0),3个模式频率对应振动。

    如果用矩阵形式来表示Hessian矩阵,应有
    \begin{equation*}
        \bm{\mathcal{H} = M}^{-\frac 12} \bm{V}^{(2)} \bm{M}^{-\frac 12}
    \end{equation*}

\section{20201023:简正坐标}

    上一节研究的Hessian矩阵对角化向量形式应为
    \begin{equation*}
        \bm{\mathcal{H}b}_j = \omega_j^2 \bm{b}_j
    \end{equation*}
    由此可知 
    \begin{equation*}
        \bm{H}
        \begin{pmatrix}
            \bm{b}_1 & \cdots & \bm{b}_N
        \end{pmatrix}
        = 
        \begin{pmatrix}
            \bm{b}_1 & \cdots & \bm{b}_N
        \end{pmatrix}
        \bm{\Omega}
    \end{equation*}
    所以
    \begin{equation*}
        \bm{T} = 
        \begin{pmatrix}
            \bm{b}_1 & \cdots & \bm{b}_N
        \end{pmatrix}
    \end{equation*}
    有
    \begin{equation*}
        \bm{HT = T\Omega}
    \end{equation*}
    要得到本征值,应当有
    \begin{equation*}
        \det (\bm{\mathcal{H}}- \omega^2 \bm{I}) = 0
    \end{equation*}
    即可得到$N$本征频率。
    \begin{asg}
        第3次作业第1题:证明Hessian矩阵的本征值都是实数。
    \end{asg}

    水分子在谐振子近似下的势能函数为
    \begin{align*}
        V(\bm{x}) &= V(\bm{x}_\mathrm{eq}) + \frac 12 (\bm{x-x}_\mathrm{eq})^\mathrm{T} \bm{V}^{(2)} (\bm{x-x}_\mathrm{eq})\\
        &= V(\bm{x}_\mathrm{eq}) + \frac 12 (\bm{x-x}_\mathrm{eq})^\mathrm{T} \bm{M}^{\frac 12}\bm{\mathcal{H}} \bm{M}^{\frac 12} (\bm{x-x}_\mathrm{eq})\\
        &= V(\bm{x}_\mathrm{eq}) + \frac 12 (\bm{x-x}_\mathrm{eq})^\mathrm{T} \bm{M}^{\frac 12} \bm{T\Omega T}^\mathrm{T} \bm{M}^{\frac 12} (\bm{x-x}_\mathrm{eq})
    \end{align*}
    定义\textbf{简正坐标}$\bm{Q}$为
    \begin{equation*}
        \bm{Q} = \bm{T}^\mathrm{T} \bm{M}^{\frac 12} (\bm{x-x}_\mathrm{eq})
    \end{equation*}
    于是势能面可以写为
    \begin{equation*}
        V(\bm{Q}) = V(\bm{0}) + \frac 12 \bm{Q}^\mathrm{T} \bm{\Omega Q} = V(\bm{0}) + \sum_{j=1}^N \frac 12 \omega_j^2 Q_j^2
    \end{equation*}

\section{20201026:简正坐标和Cartesian坐标的关系}

    上一节讨论了势能在简正坐标下的形式。要想得到全能量,还需要给出动量在简正坐标下的形式。根据
    \begin{equation*}
        \bm{p} = \bm{M\dot{x}} = \bm{M}^{\frac 12} \bm{T\dot{Q}}
    \end{equation*}
    这是在Cartesian坐标系下的动量。定义简正坐标下的动量为
    \begin{equation*}
        \bm{P = \dot{Q}} = \bm{T}^\mathrm{T} \bm{M}^{-\frac 12} \bm{p}
    \end{equation*}
    那么,可以得到动能的表达式为
    \begin{align*}
        E_\mathrm{k} = \frac 12 \bm{p}^{\mathrm{T}} \bm{M}^{-1} \bm{p} &= \frac 12 \bm{\dot{x}}^\mathrm{T}\bm{M\dot{x}}
        = \frac 12 \bm{\dot{Q}}^\mathrm{T} \bm{Q}
        = \frac 12 \bm{P}^{\mathrm{T}} \bm{P}
    \end{align*}
    总结简正坐标和Cartesian坐标的变换:
    \begin{align*}
        \bm{Q} &= \bm{T}^\mathrm{T} \bm{M}^{\frac 12} (\bm{x-x}_\mathrm{eq})\\
        \bm{P} &= \bm{T}^\mathrm{T} \bm{M}^{-\frac 12} \bm{p}\\
        \bm{x} &= \bm{x}_\mathrm{eq} + \bm{M}^{-\frac 12}\bm{TQ}\\
        \bm{p} &= \bm{M}^{\frac 12}\bm{TP}
    \end{align*}

    有了简正坐标下的动量就可以得到简正坐标下的Hamilton函数:
    \begin{equation*}
        H = \frac 12 \bm{P}^\mathrm{T}\bm{P} + \frac 12 \bm{Q}^\mathrm{T} \bm{\Omega Q}
    \end{equation*}
    很容易可以验证,正则方程在简正坐标下依旧成立:
    \begin{align*}
        \bm{\dot{Q}} &= \frac {\partial H}{\partial \bm{P}} = \bm{P}\\
        \bm{\dot{P}} &= -\frac {\partial H}{\partial \bm{Q}} = -\bm{\Omega Q}
    \end{align*}
    要求出每个元素的值也十分容易:
    \begin{align*}
        \dot{Q}_j &= P_j\\
        \dot{P}_j &= - \omega_j^2 Q_j
    \end{align*}
    但如果在Cartesian坐标下用正则方程,得到每个元素的值结果为
    \begin{align*}
        \dot{x}_i &= \frac {p_j}{m_i}\\
        \dot{p}_i &= \sum_j \frac {\partial^2 V}{\partial x_i \partial x_j} (x_j - x_\mathrm{eq}^{(j)})
    \end{align*}
    显然要比在简正坐标下的形式要复杂很多。这体现了简正坐标的优势。

    我们可以根据初始条件$(\bm{x}_0,\bm{p}_0)$,得到$(\bm{x},\bm{p}$每个分量的解析表达式。将它变换为简正坐标,得到
    \begin{align*}
        Q_j &= \sum_i T_{ij} m_i^{\frac 12} (x_0^{(i)} - x_\mathrm{eq}^{(i)})\\
        P_j &= \sum_i T_{ij} m_i^{-\frac 12} (p_0^{(i)})
    \end{align*}
    由正则方程的形式可以给出
    \begin{align*}
        Q_j(t) &= Q_0^{(j)}\cos{\omega_j t} + \frac {P_0^{(j)}}{\omega} \sin{\omega_j t}\\
        P_j(t) &= P_0^{(j)}\cos{\omega_j t} - \omega_j Q_0^{(j)} \sin{\omega_j t}
    \end{align*}
    再变换回Cartesian坐标,得到
    \begin{align*}
        x_i(t) - x_\mathrm{eq}^{(i)} &= \sum_j m_j^{-\frac 12} T_{ij} Q_j(t)\\
        p_i(t) &= \sum_j m_j^{\frac 12} T_{ij}P_j(t)
    \end{align*}

    如果水分子服从Boltzmann分布,即
    \begin{align*}
        \rho \propto \mathrm{e}^{-\beta (\frac 12 \bm{P}^\mathrm{T}\bm{P} + \frac 12 \bm{Q}^\mathrm{T} \bm{\Omega Q})} = \mathrm{e}^{- \frac {\beta}2 \sum_j (P_j^2 + \omega_j^2 Q_j^2)} = \prod_j \mathrm{e}^{-\frac {\beta}2 (P_j^2 + \omega_j^2 Q_j^2)}
    \end{align*}
    要想将Cartesian坐标下的积分变换成简正坐标下的积分,需要计算Jacobi行列式:
    \begin{align*}
        \bigg|\frac {\partial (\bm{Q,P})}{\partial (\bm{x,p})}\bigg| &= \det
        \begin{pmatrix}
            \bm{T}^\mathrm{T}\bm{M}^{\frac 12} & 0\\
            0 & \bm{T}^\mathrm{T}\bm{M}^{-\frac 12}
        \end{pmatrix}
        = 1
    \end{align*}
    这里用到了正交矩阵的行列式为1(这也是显然成立的)。因此可以积分得到配分函数,从而得到简正坐标下的某个分量概率密度为
    \begin{equation*}
        \mathcal{P}_j = \frac {2\pi}{\beta\omega_j} \mathrm{e}^{-\frac {\beta}2 (P_j^2 + \omega_j^2 Q_j^2)}
    \end{equation*}
    总概率密度为
    \begin{equation*}
        \mathcal{P} = \prod_j \mathcal{P}_j
    \end{equation*}
    由此可以得到Cartesian坐标下概率密度。

\section{20201030:物理量及其时间关联函数}

    对简正坐标下的Hamilton函数
    \begin{equation*}
        H = \frac 12 \bm{P}^\mathrm{T}\bm{P} + \frac 12 \bm{Q}^\mathrm{T} \bm{\Omega Q}
    \end{equation*}
    它满足Boltzmann分布时,配分函数为
    \begin{equation*}
        Z = \int \mathrm{e}^{-\beta H} \mathrm{d}\bm{Q}\mathrm{d}\bm{P} = \bigg(\frac {2\pi}{\beta}\bigg)^N \frac 1{\det \bm{\Omega}}
    \end{equation*}
    量子化以后得到的结果是
    \begin{equation*}
        Z = \frac 1{(\beta \hbar)^N \det \bm{\Omega}}
    \end{equation*}
    要计算物理量的期望,应有
    \begin{equation*}
        \langle B \rangle = \frac {\int B(\bm{Q,P})\mathrm{e}^{-\beta H} \mathrm{d}\bm{Q}\mathrm{d}\bm{P}}{\int \mathrm{e}^{-\beta H} \mathrm{d}\bm{Q}\mathrm{d}\bm{P}}
    \end{equation*}
    在$t$时刻也可以写出类似的形式
    \begin{equation*}
        \langle B(t) \rangle = \frac {\int B(\bm{Q,P}) \rho_t(\bm{Q,P}) \mathrm{d}\bm{Q}\mathrm{d}\bm{P}}{\int \rho_t(\bm{Q,P}) \mathrm{d}\bm{Q}\mathrm{d}\bm{P}}
    \end{equation*}
    再由
    \begin{equation*}
        \rho_t (\bm{Q,P}) = \int \mathrm{e}^{-\beta H(\bm{Q_0,P_0})} \delta(\bm{Q-Q}_t) \delta(\bm{P-P}_t) \mathrm{d}\bm{Q}_0\mathrm{d}\bm{P}_0
    \end{equation*}
    代入,可以得到
    \begin{equation*}
        \langle B(t) \rangle = \frac {\int B(\bm{Q}_t,\bm{P}_t) \rho_0(\bm{Q}_0,\bm{P}_0) \mathrm{d}\bm{Q}_0\mathrm{d}\bm{P}_0}{\int \rho_0(\bm{Q}_0,\bm{P}_0) \mathrm{d}\bm{Q}_0\mathrm{d}\bm{P}_0}
    \end{equation*}

    现在开始研究一些光谱的性质。设红外光谱为$I(\omega)$, 让分子不转动,则得到的红外光谱为分立的线。对红外光谱做Fourier变换,得到
    \begin{equation*}
    f(t) = \int I(\omega)\mathrm{e}^{\mathrm{i}\omega t}\mathrm{d}\omega 
    \end{equation*}
    它反映了分子的动力学性质。这个时间是什么?现在问有没有可能成为某个物理量的Fourier变换?
    \begin{asg}
        第4次作业第1题:Fourier变换
    \end{asg}
    定义\textbf{两点时间关联函数}:
    \begin{equation*}
    \langle B(0)B(t) \rangle = \int \rho_0(x_0,p_0) B(x_0,p_0) B(x_t(x_0,p_0),p_t(x_0,p_0)) \mathrm{d}x_0 \mathrm{d}p_0
    \end{equation*}
    同一物理量的两点时间关联函数如果交换顺序并不会有变化,因为Liouville定理,
    \begin{align*}
    \langle B(0)B(t) \rangle &= \int \rho_0(x_0,p_0) B(x_0,p_0) B(x_t(x_0,p_0),p_t(x_0,p_0)) \mathrm{d}x_0 \mathrm{d}p_0\\
    &= \int \rho_t(x_t,p_t) B(x_t,p_t) B(x_0(x_t,p_t),p_0(x_t,p_t)) \mathrm{d}x_t \mathrm{d}p_t\\
    &= \langle B(t)B(0) \rangle
    \end{align*}
    现在要求$\langle B(0)B(t) \rangle$和$\langle B(0)B(-t) \rangle$的关系。在积分的条件下,因为积分变量是哑变量,
    \begin{align*}
    \langle B(0)B(t) \rangle &= \int \rho_0(x_0,p_0) B(x_0,p_0) B(x_t(x_0,p_0),p_t(x_0,p_0)) \mathrm{d}x_0 \mathrm{d}p_0\\
    &= \int \rho_0(x_{-t},p_{-t}) B(x_{-t},p_{-t}) B(x_0(x_{-t},p_{-t}),p_0(x_{-t},p_{-t})) \mathrm{d}x_{-t} \mathrm{d}p_{-t}\\
    &= \int \rho_0(x_{-t},p_{-t}) B(x_{-t},p_{-t}) B(x_0(x_{-t},p_{-t}),p_0(x_{-t},p_{-t})) \mathrm{d}x_{-t} \mathrm{d}p_{-t}\\
    \end{align*}
    其中,第二步是作变量替换
    \begin{equation*}
    x_0 \to x_{-t}
    \end{equation*}
    并且$x_t(x_0,p_0)$是初始时间为0时演化$t$时间的结果,而将$x_{-t}$演化$t$时间为$x_0$。如果假设
    \begin{equation*}
    \frac {\partial \rho}{\partial t} = 0 = \{ H, \rho\}
    \end{equation*}
    则显然地,
    \begin{align*}
        \langle B(0)B(t) \rangle &= \int \rho_0(x_{-t},p_{-t}) B(x_{-t},p_{-t}) B(x_0(x_{-t},p_{-t}),p_0(x_{-t},p_{-t})) \mathrm{d}x_{-t} \mathrm{d}p_{-t}\\
        &= \int \rho_{-t}(x_{-t},p_{-t}) B(x_{-t},p_{-t}) B(x_0(x_{-t},p_{-t}),p_0(x_{-t},p_{-t})) \mathrm{d}x_{-t} \mathrm{d}p_{-t}\\
        &= \langle B(-t)B(0) \rangle
    \end{align*}
    \begin{asg}
        第4次作业第2题:时间自关联函数是否有时间平移对称性?
    \end{asg}

\section{20201102:平衡分布的时间关联函数}
    更一般情况的两点时间关联函数为
    \begin{equation*}
    \langle A(0)B(t) \rangle = \int \rho_0(\bm{x}_0,\bm{p}_0) A(\bm{x}_0,\bm{p}_0) B(\bm{x}_t,\bm{p}_t) \mathrm{d}\bm{x}_0 \mathrm{d}\bm{p}_0
    \end{equation*}
    如果是平衡分布,即
    \begin{equation*}
        \frac {\partial \rho}{\partial t} = 0
    \end{equation*}
    例如Boltzmann分布,那么
    \begin{align*}
        \langle A(0)B(t) \rangle &= \int \rho_\mathrm{eq}(\bm{x}_0,\bm{p}_0) A(\bm{x}_0,\bm{p}_0) B(\bm{x}_t,\bm{p}_t) \mathrm{d}\bm{x}_0 \mathrm{d}\bm{p}_0\\
        &= \int \rho_\mathrm{eq}(\bm{x}_{t'},\bm{p}_{t'}) A(\bm{x}_{t'},\bm{p}_{t'}) B(\bm{x}_{t+t'},\bm{p}_{t+t'}) \mathrm{d}\bm{x}_{t'} \mathrm{d}\bm{p}_{t'}\\
        &= \langle A(t')B(t'+ t) \rangle
    \end{align*}
    这样时间关联函数有时间平移对称性。但是如果不是平衡分布,就没有时间平移对称性。同样由Liouville定理很容易证明
    \begin{equation*}
        \langle A(0)B(t) \rangle = \langle B(t)A(0) \rangle
    \end{equation*}

    平衡分布满足
    \begin{equation*}
    \langle B(t) \rangle = \langle B(0) \rangle
    \end{equation*}
    于是平衡分布对应的平均物理量为
    \begin{equation*}
    \langle B \rangle = \frac 1T \int_0^T \langle B(t) \rangle \mathrm{d}t
    \end{equation*}

    对于Liouville方程,有
    \begin{equation*}
    -\frac {\partial \rho}{\partial t} = \{H,\rho\}
    \end{equation*}
    此时,满足
    \begin{equation*}
    \rho_0 = \rho_{\mathrm{eq}}
    \end{equation*}
    两点关联函数满足
    \begin{align*}
    \langle A(0)B(t) \rangle &= \int \rho_{\mathrm{eq}} (\bm{x_0},\bm{p_0}) A(\bm{x_0},\bm{p_0}) B(\bm{x_t}, \bm{p_t})\\
    &= \int \rho_{\mathrm{eq}} (\bm{x_{t'}},\bm{p_{t'}}) A(\bm{x_{t'}},\bm{p_{t'}}) B(\bm{x_{t+t'}}, \bm{p_{t + t'}})\\
    &= \langle A(t')B(t+t') \rangle
    \end{align*}
    这只有在
    \begin{equation*}
    \rho_0(\bm{x},\bm{p}) = \rho_{t'}(\bm{x},\bm{p})
    \end{equation*}
    时成立。

    现在想要探索$\langle A(0)B(t) \rangle$和$\langle A(0)B(-t) \rangle$的关系。
    \begin{align*}
    \langle A(0)B(t) \rangle &= \int \rho_{\mathrm{eq}} (\bm{x_0},\bm{p_0}) A(\bm{x_0},\bm{p_0}) B(\bm{x_t}, \bm{p_t})\\
    &= \int \rho_{\mathrm{eq}}(\bm x_t, \bm p_t) A(\bm x_0,\bm p_0)B(\bm x_t, \bm p_t) \mathrm{d}\bm x_t \mathrm{d} \bm p_t\\
    &= \int \rho_{\mathrm{eq}}(\bm x, \bm p)B(\bm x, \bm p) A(\bm x_{-t}(x,p), \bm p_{-t}(x,p))\mathrm{d}x\mathrm{d}p\\
    &= \langle B(0)A(-t) \rangle
    \end{align*}
    如果只考虑一个物理量的自关联函数,作Fourier积分
    \begin{align*}
    I(\omega) = \int_{-\infty}^{+\infty} \mathrm{e}^{-\mathrm{i}\omega t} \langle B(0)B(t) \rangle \mathrm{d}t
    \end{align*}
    令$t=-s$,则
    \begin{align*}
    I(\omega) &= -\int_{+\infty}^{-\infty} \mathrm{e}^{\mathrm{i}\omega s} \langle B(0)B(-s) \rangle \mathrm{d}s\\
    &= \int_{-\infty}^{+\infty} \mathrm{e}^{\mathrm{i}\omega s} \langle B(0)B(-s) \rangle \mathrm{d}s\\
    &= \int_{-\infty}^{+\infty} \mathrm{e}^{\mathrm{i}\omega t} \langle B(0)B(-t) \rangle \mathrm{d}t\\
    &= \int_{-\infty}^{+\infty} \mathrm{e}^{\mathrm{i}\omega t} \langle B(0)B(t) \rangle \mathrm{d}t\\
    &= I(-\omega)
    \end{align*}
    所以自关联函数的Fourier变换在频率空间是一个偶函数。

    这在量子力学中并不成立,如果它在能级0和能级1之间跃迁,则它在能级0的概率为$\frac 1Z \mathrm{e}^{-\beta \epsilon_0}$,在能级1的概率为$\frac 1Z \mathrm{e}^{-\beta \epsilon_1}$, 配分函数为
    \begin{equation*}
    Z = \mathrm{e}^{-\beta \epsilon_0} + \mathrm{e}^{-\beta \epsilon_1}
    \end{equation*}
    设$E = \epsilon_1 - \epsilon_0$, 从0到1的跃迁对应的光谱Fourier变换的强度为$\mathrm{e}^{-\beta \epsilon_0} \delta(E - \hbar \omega)$,从1到0的跃迁对应的强度为$\mathrm{e}^{-\beta \epsilon_1} \delta(E + \hbar \omega)$, 于是
    \begin{equation*}
    \mathrm{e}^{-\beta(\epsilon_1 - \epsilon_0)} I(\omega) = I(-\omega)
    \end{equation*}
    或写成
    \begin{equation*}
        \mathrm{e}^{-\beta \hbar \omega}I(\omega) = I(-\omega)
    \end{equation*}
    这与前面所得到的经典情况下Fourier变换得到的频谱为偶函数的结论并不相同,称为\textbf{细致平衡}。经典极限下,$\hbar \to 0$,变成了偶函数。

\section{20201106:Gauss积分的计算}

    回顾一维Gauss积分的计算:
    \begin{align*}
        I &= \int_0^{+\infty} \mathrm{e}^{-ax^2} x^{n} \mathrm{d}x
    \end{align*}
    令$t = ax^2$, 则$\mathrm{d}t = 2ax\mathrm{d}x$
    所以
    \begin{align*}
    I &= \int_0^{+\infty} \mathrm{e}^{-t} \bigg(\frac ta\bigg)^{\frac n2} \frac {\mathrm{d}t}{\sqrt{at}}\\
    &= \frac 1{2a^{\frac {n+1}2}} \int_0^{+\infty} \mathrm{e}^{-t} t^{\frac {n-1}2} \mathrm{d}t\\
    &= \frac {\Gamma(\frac {n+1}2)}{2a^{\frac {n+1}2}}
    \end{align*}

    利用一维Gauss积分的计算结果可以计算多维的Gauss积分。比如计算
    \begin{equation*}
        I = \int \bm{x}^\mathrm{T}\bm{Bx} \mathrm{e}^{-\bm{x}^\mathrm{T}\bm{Ax}}\mathrm{d}\bm{x}
    \end{equation*}
    其中,$\bm{A}$为正定的实对称矩阵。首先将$\bm{A}$对角化,得到
    \begin{equation*}
        \bm{T}^\mathrm{T} \bm{AT} = \bm{D}
    \end{equation*}
    其中$\bm{T}$为正交矩阵、$\bm{D}$为对角矩阵。所以
    \begin{equation*}
        I = \int \bm{x}^\mathrm{T}\bm{Bx} \mathrm{e}^{-\bm{x}^\mathrm{T}\bm{TDT}^\mathrm{T}\bm{x}}\mathrm{d}\bm{x}
    \end{equation*}
    作换元$\bm{y} = \bm{T}^\mathrm{T}\bm{x}$,则有
    \begin{equation*}
        \mathrm{d}\bm{y} = \det \bm{T}^\mathrm{T}\mathrm{d}\bm{x} = \mathrm{d}\bm{x}
    \end{equation*}
    原积分化为
    \begin{align*}
        I = \int \bm{y}^\mathrm{T}\bm{T}^\mathrm{T}\bm{BTy} \mathrm{e}^{-\bm{y}^\mathrm{T}\bm{D}\bm{y}}\mathrm{d}\bm{y}
    \end{align*}
    令$\bm{E = T}^\mathrm{T}\bm{BT}$, 于是
    \begin{align*}
        I &= \int \bm{y}^\mathrm{T}\bm{Ey} \mathrm{e}^{-\bm{y}^\mathrm{T}\bm{D}\bm{y}}\mathrm{d}\bm{y}\\
        &= \int \sum_i \sum_j E_{ij} y_i y_j \mathrm{e}^{-\sum_k d_k y_k^2} \prod_k \mathrm{d}y_k
    \end{align*}
    可以通过分离变量将各个积分分开,显然,$i \neq j$的项都是奇函数对全空间的积分,得到的结果为0,只有$i=j$的项会有贡献。于是积分化为
    \begin{align*}
        I &= \int \sum_i E_{ii} y_i^2 \mathrm{e}^{-\sum_k d_k y_k^2} \prod_k \mathrm{d}y_k\\
        &= \prod_k \sqrt{\frac {\pi}{d_k}} \sum_i \frac {E_{ii}}{2d_i}\\
        &= \frac {\pi^{\frac n2}}{2\sqrt{\det{\bm{D}}}} \mathrm{Tr} (\bm{ED}^{-1})
    \end{align*}
    而
    \begin{align*}
        \det{\bm{D}} = \det{\bm{A}}
    \end{align*}
    并且
    \begin{align*}
        \mathrm{Tr}(\bm{ED}^{-1}) = \mathrm{Tr}(\bm{T}^\mathrm{T}\bm{BT}\bm{T}^\mathrm{T} \bm{A}^{-1}\bm{T})
        = \mathrm{Tr}(\bm{T}^\mathrm{T}\bm{BA}^{-1}\bm{T})
        = \mathrm{Tr}(\bm{BA}^{-1})
    \end{align*}
    所以
    \begin{align*}
        I = \frac {\pi^{\frac n2}}{2\sqrt{\det{\bm{A}}}} \mathrm{Tr} (\bm{BA}^{-1})
    \end{align*}

    接下来尝试计算
    \begin{align*}
        \bm{I} = \int \bm{xx}^\mathrm{T} \mathrm{e}^{-\bm{x}^\mathrm{T}\bm{Ax}}\mathrm{d}\bm{x}
    \end{align*}
    用相同的还原方法得到
    \begin{align*}
        \bm{I} = \int \bm{Ty}\bm{y}^\mathrm{T}\bm{T}^\mathrm{T} \mathrm{e}^{-\bm{y}^\mathrm{T}\bm{Dy}}\mathrm{d}\bm{y}
    \end{align*}
    计算每个元素
    \begin{align*}
        I_{ij} &= \int (\bm{Ty})_{i}(\bm{y}^\mathrm{T}\bm{T}^\mathrm{T})_{j} \mathrm{e}^{-\bm{y}^\mathrm{T}\bm{Dy}}\mathrm{d}\bm{y}\\
        &= \int \sum_k T_{ik}y_k \sum_l T_{jl} \mathrm{e}^{-\bm{y}^\mathrm{T}\bm{Dy}}\mathrm{d}\bm{y}\\
        &= \int \sum_{k,l} T_{ik}T_{jl} y_ky_l \mathrm{e}^{-\bm{y}^\mathrm{T}\bm{Dy}}\mathrm{d}\bm{y}
    \end{align*}
    同样地,只有在$k=l$时才有贡献,故
    \begin{align*}
        I_{ij} &= \int \sum_k T_{ik}T_{jk}y_k^2 \mathrm{e}^{-\bm{y}^\mathrm{T}\bm{Dy}}\mathrm{d}\bm{y}\\
        &= \prod_l \sqrt{\frac {\pi}{d_l}} \sum_k T_{ik}T_{jk} \frac 1{2d_k}\\
        &= \frac {\pi^{\frac n2}}{2 \sqrt{\det{\bm{D}}}} (\bm{TD}^{-1}\bm{T}^\mathrm{T})_{ij}\\
        &= \frac {\pi^{\frac n2}}{2 \sqrt{\det{\bm{A}}}} \bm{A}^{-1}_{ij}
    \end{align*}
    因此,
    \begin{align*}
        \bm{I} = \frac {\pi^{\frac n2}}{2 \sqrt{\det{\bm{A}}}} \bm{A}^{-1}
    \end{align*}
    \begin{asg}
        第4次作业第3题:Gauss积分的计算
    \end{asg}

\section{20201109:量子力学基本假设}

    从本节开始讨论量子力学。量子力学中第一个重要的概念是\textbf{态}。有了态我们可以进行测量,得到这个态的物理量。用$| \psi \rangle$来表示态。

    如何来描述这个态呢?我们可以选择一个空间进行描述。对于经典力学,我们之前选择了相空间。但对于量子力学,我们选择位置空间或者动量空间进行描述。如果选取位置空间,对这个态的描述为
    \begin{equation*}
        \langle \bm{x} | \psi \rangle = \psi(\bm{x})
    \end{equation*}
    将这个函数称为\textbf{波函数}。如果选取动量空间,类似地可以描述为
    \begin{equation*}
        \langle \bm{p} | \psi \rangle = \psi(\bm{p})
    \end{equation*}
    量子力学中,位置空间和动量空间都是连续的。

    我们也可以在离散的空间中描述态。态可以看作一个向量,它可以用一组基展开。回顾在线性代数中,
    \begin{equation*}
        \bm{c} = \sum_n c_n \bm{e}_n
    \end{equation*}
    如果这组基是内积空间中的规范正交基,则
    \begin{equation*}
        \bm{c} = \sum_n \bm{e}_n (\bm{e}_n^\mathrm{T}\bm{c}) 
        = \sum_n |n\rangle \langle n|c\rangle
    \end{equation*}
    由此可见
    \begin{equation*}
        \bm{I} = \sum_n |n\rangle \langle n|
    \end{equation*}
    在量子力学中,我们可以类似地描述态:
    \begin{equation*}
        |\psi \rangle = \sum_n |n\rangle \langle n|\psi \rangle = \sum_n c_n |n\rangle
    \end{equation*}

    物理量测量都是实数。物理量在量子力学中都对应一个算符,假设
    \begin{equation*}
        \hat{A} |n\rangle = a_n |n\rangle
    \end{equation*}
    其中$a_n$为实数,那么$|n\rangle$就是$\hat{A}$的一个本征态。我们可以把态对于$\hat{A}$的本征态来展开,得到
    \begin{align*}
        \hat{A}|\psi \rangle &= \hat{A}\hat{I}|\psi \rangle
        = \hat{A} \sum_n |n\rangle \langle n|\psi\rangle
        = \sum_n c_n a_n |n\rangle
    \end{align*}
    可以计算出$\hat{A}$的平均值为
    \begin{align*}
        \langle \hat{A} \rangle &= \langle \psi |\hat{A}| \psi \rangle\\
        &= \sum_n \langle n| c_n^* \sum_m c_m a_m |m\rangle\\
        &= \sum_n \sum_m c_n^* c_m a_m \langle n|m\rangle\\
        &= \sum_n |c_n|^2 a_n
    \end{align*}
    这里用到了
    \[\langle n | m \rangle = \delta_{nm}\]
    注意到$|c_n|^2 \in [0,1]$,且$\sum_n |c_n|^2 = 1$(我们总可以让这个态乘一个常数使该式成立,此时态也满足归一化条件$\langle \psi |\psi \rangle = 1$), 所以可以认为这个态处于该本征态的概率。按照Copenhagen学派的观点,我们测量某一个物理量时这个态会坍塌到这个物理量的一个本征态,而$|c_n^2|$反应了坍塌到第$n$个本征态的概率。

    对于波函数$\langle \bm{x}|\psi \rangle$, 它的模方是在位置空间的概率密度。定义一个位置算符$\hat{\bm{y}}$。应有
    \begin{equation*}
        \hat{\bm{x}}|\bm{x}_0\rangle = \bm{x}_0 |\bm{x}_0\rangle
    \end{equation*}
    其中态$|\bm{x}_0\rangle$代表精确地处在$\bm{x}_0$位置的态。引入动量算符$\hat{\bm{p}}$,同样有
    \begin{equation*}
        \hat{\bm{p}}|\bm{p}_0\rangle = \bm{p}_0 |\bm{p}_0\rangle
    \end{equation*}
    类比在可数个物理量本征态下的展开,同样有
    \begin{equation*}
        \hat{I} = \int |\bm{x}\rangle \langle \bm{x}| \mathrm{d}\bm{x}
    \end{equation*}
    于是对位置的测量应有
    \begin{equation*}
        \hat{\bm{x}}|\psi \rangle = \int \hat{\bm{x}}|\bm{x}\rangle \langle \bm{x}|\psi \rangle \mathrm{d}\bm{x}
        = \int \bm{x}|\bm{x}\rangle \langle \bm{x}|\psi \rangle \mathrm{d}\bm{x}
    \end{equation*}
    同样,任意一个态可以展开到位置空间
    \begin{equation*}
        |\psi \rangle = \int |\bm{x} \rangle \langle \bm{x}|\psi\rangle \mathrm{d}\bm{x}
    \end{equation*}
    类似地得到位置的平均值为
    \begin{equation*}
        \langle \psi | \hat{\bm{x}} | \psi \rangle 
        = \int \bm{x} |\langle \bm{x} | \psi \rangle|^2 \mathrm{d}\bm{x}
    \end{equation*}
    这就给出了波函数的概率诠释。

    接下来讨论动量算符在位置空间的描述。假设
    \begin{equation*}
        \langle \bm{x} |\hat{\bm{p}} | \psi \rangle
        = \langle \bm{x} | \phi \rangle
    \end{equation*}
    如果选择
    \begin{equation*}
        \hat{\bm{p}} = -\mathrm{i}\hbar \frac {\partial }{\partial \bm{x}}
    \end{equation*}
    就得到
    \begin{equation*}
        \langle \bm{x} |\hat{\bm{p}} | \psi \rangle = -\mathrm{i}\hbar \frac {\partial }{\partial \bm{x}}\langle \bm{x} | \psi \rangle
    \end{equation*}
    有了位置算符和动量算符,我们可以讨论两个算符的\textbf{对易}
    \begin{equation*}
        [\hat{\bm{x}},\hat{\bm{p}}] = \hat{\bm{x}}\hat{\bm{p}} - \hat{\bm{p}}\hat{\bm{x}}
    \end{equation*}
    先计算
    \begin{align*}
        \langle \bm{x}_0 |\hat{\bm{p}}\hat{\bm{x}} | \psi \rangle &= -\mathrm{i}\hbar \frac {\partial}{\partial \bm{x}_0} \langle \bm{x}_0 |\hat{\bm{x}} | \psi \rangle\\
        &= -\mathrm{i}\hbar \frac {\partial}{\partial \bm{x}_0} (\bm{x}_0\psi(\bm{x}_0))\\
        &= -\mathrm{i}\hbar \psi(\bm{x}_0) - \mathrm{i}\hbar \bm{x}_0 \frac {\partial \psi(\bm{x}_0)}{\partial \bm{x}_0}
    \end{align*}
    再计算
    \begin{align*}
        \langle \bm{x}_0 |\hat{\bm{x}} \hat{\bm{p}} | \psi \rangle &= \bm{x}_0 \langle \bm{x}_0 \hat{\bm{p}} | \psi \rangle\\
        &= -\mathrm{i}\hbar \bm{x}_0 \frac {\partial \psi(\bm{x}_0)}{\partial \bm{x}_0}
    \end{align*}
    这两个式子相比较,得到
    \begin{equation*}
        [\hat{\bm{x}},\hat{\bm{p}}] = \mathrm{i}\hbar
    \end{equation*}
    此即\textbf{Heisenberg不确定性原理}。

\section{20201113:不确定性原理}

    总结一下量子力学的基本假设
    \begin{enumerate}
        \item 波函数:态$|\psi \rangle$可以在位置空间描述$\langle x|\psi\rangle$
        \item 算符:物理量对应Hermite算符。
        \item 测量:对某个态测量某个物理量,会得到其本征值。
        \item 不确定性原理:$[\hat{\bm{x}},\hat{\bm{p}}] = \mathrm{i}\hbar$
    \end{enumerate}
    在量子力学中,位置空间、动量空间是连续的,时间也是连续的,并且认为质量不变。量子力学中,位置和动量都有对应的算符,但时间没有。

    可以定义位置的量子涨落
    \begin{equation*}
        \Delta x = \sqrt{\langle \hat{x}^2 \rangle - \langle \hat{x} \rangle^2}
    \end{equation*}
    同理可以定义动量的量子涨落
    \begin{equation*}
        \Delta p = \sqrt{\langle \hat{p}^2 \rangle - \langle \hat{p} \rangle^2}
    \end{equation*}
    现在定义 
    \begin{align*}
        \Delta \hat{x} &= \hat{x} - \langle \hat{x} \rangle \\
        \Delta \hat{p} &= \hat{p} - \langle \hat{p} \rangle
    \end{align*}
    希望求出
    \begin{align*}
        \langle \Delta \hat{x}^2\rangle\langle \Delta \hat{p}^2\rangle
    \end{align*}
    设
    \begin{align*}
        |\phi_x \rangle &= \Delta \hat{x}^2|\psi\rangle\\
        |\phi_p \rangle &= \Delta \hat{p}^2|\psi\rangle
    \end{align*}
    于是 
    \begin{align*}
        \langle \Delta \hat{x}^2\rangle\langle \Delta \hat{p}^2\rangle = \langle \phi_x | \phi_x \rangle \langle \phi_p | \phi_p \rangle
    \end{align*}
    考察
    \begin{align*}
        |\langle \phi_x | \phi_p \rangle| = |\bm{a}^\dagger \bm{b}| = |\bm{a}||\bm{b}|\cos{\theta} \leqslant |\bm{a}||\bm{b}|
    \end{align*}
    应有
    \begin{align*}
        \langle \Delta \hat{x}^2\rangle\langle \Delta \hat{p}^2\rangle &= \langle \phi_x | \phi_x \rangle \langle \phi_p | \phi_p \rangle\\ &\geqslant |\langle \phi_x|\phi_p \rangle|^2\\
        &= |\langle \psi |\Delta \hat{x} \Delta \hat{p}|\psi \rangle|^2
    \end{align*}
    所以只需要求出
    \begin{align*}
        \Delta \hat{x} \Delta \hat{p} = \frac 12([\Delta \hat{x}, \Delta \hat{p}] + \{\Delta \hat{x}, \Delta \hat{p}\})
    \end{align*}
    其中反对易关系
    \begin{equation*}
        \{\hat{A}, \hat{B}\} = \hat{A}\hat{B} + \hat{B}\hat{A}
    \end{equation*}
    因此
    \begin{equation*}
        |\langle \psi |\Delta \hat{x} \Delta \hat{p}|\psi \rangle|^2 = \frac 14 |\langle \psi |[\Delta \hat{x},\Delta \hat{p}]|\psi \rangle + \langle \psi |\{\Delta \hat{x},\Delta \hat{p}\}|\psi \rangle|^2
    \end{equation*}
    注意
    \begin{equation*}
        [\Delta \hat{x},\Delta \hat{p}] = [\hat{x}-\langle \hat{x} \rangle, \hat{p}-\langle \hat{p} \rangle ] = [\hat{x},\hat{p}] = \mathrm{i}\hbar
    \end{equation*}
    上面就是一个复数模的平方,得到
    \begin{equation*}
        |\langle \psi |\Delta \hat{x} \Delta \hat{p}|\psi \rangle|^2 = \frac {\hbar^2}4 + \frac 14 |\langle \psi |\{\Delta \hat{x},\Delta \hat{p}\}|\psi \rangle|^2 \geqslant \frac {\hbar^2}4
    \end{equation*}
    这是不确定性原理的另一种表述形式。

    现在来在位置空间描述动量本征态,即求出$\langle x|p \rangle$。这表示动量精确地处在$p$时,在位置空间的描述。显然地,它满足
    \begin{equation*}
        \langle x|\hat{p}|p\rangle = p\langle x|p \rangle
    \end{equation*}
    由此可知 
    \begin{align*}
        -\mathrm{i}\hbar \frac {\partial}{\partial x} \langle x|p \rangle = p \langle x|p \rangle
    \end{align*}
    解这个常微分方程,得到
    \begin{equation*}
        \langle x|p \rangle = C \mathrm{e}^{\frac {\mathrm{i}px}\hbar}
    \end{equation*}
    $C$由归一化条件决定。首先考虑
    \begin{equation*}
        \langle p' | p_0 \rangle = \delta (p'-p_0)
    \end{equation*}
    这是因为
    \begin{equation*}
        |p_0 \rangle = \int |p\rangle \langle p|p_0\rangle \mathrm{d}p
    \end{equation*}
    显然$\delta$函数满足这个要求。又
    \begin{equation*}
        \langle p' | p_0 \rangle = \delta (p'-p_0) = \frac 1{2\pi\hbar} \int \mathrm{e}^{\frac {\mathrm{i}(p-p_0)x}{\hbar}} \mathrm{d}x
    \end{equation*}
    并且
    \begin{align*}
        \langle p'|p_0 \rangle &= \int \langle p'|x\rangle \langle x|p_0 \rangle \mathrm{d}x\\
        &= \int (\langle x|p' \rangle)^* \langle x|p_0 \rangle \mathrm{d}x\\
        &= C^*C\int \mathrm{e}^{\frac {\mathrm{i}(p_0-p')x}{\hbar}} \mathrm{d}x
    \end{align*}
    所以, 
    \begin{equation*}
        C = \frac 1{\sqrt{2\pi\hbar}}
    \end{equation*}
    这样就得到 
    \begin{equation*}
        \langle x|p \rangle = \frac 1{\sqrt{2\pi\hbar}} \mathrm{e}^{\frac {\mathrm{i}px}{\hbar}}
    \end{equation*}
    并由此可以得到
    \begin{equation*}
        \langle p|x \rangle = \frac 1{\sqrt{2\pi\hbar}} \mathrm{e}^{-\frac {\mathrm{i}px}{\hbar}}
    \end{equation*}
    \begin{asg}
        第5次作业第1题:计算动量空间的位置算符。
    \end{asg}
    \begin{asg}
        第5次作业第2题:$\delta$函数算符问题。
    \end{asg}

\section{20201116:一维无限深势阱}

    考虑动能算符和动量算符的对易关系,
    \begin{equation*}
        [\frac {\hat{p}^2}{2m}, \hat{p}] = 0
    \end{equation*}
    事实上可以证明
    \begin{equation*}
        [f(\hat{A}),g(\hat{A})] = 0
    \end{equation*}
    \begin{asg}
        第6次作业第1题(1):证明上述结论。
    \end{asg}
    如果
    \[ [\hat{A},\hat{B}]=0 \]
    并设$|\phi_n\rangle$是$\hat{A}$的一个本征态
    \[ \hat{A} |\phi_n \rangle = a_n |\phi_n \rangle \]
    那么
    \[ \hat{A} \hat{B} |\phi_n \rangle = \hat{B}\hat{A} |\phi_n \rangle = a_n \hat{B} |\phi_n \rangle \]
    这说明,如果$\hat{A},\hat{B}$对易,则$\hat{B}|\phi_n\rangle$必然是$\hat{A}$的本征态,且本征值为$a_n$。如果不简并,那么$\hat{B}|\phi_n\rangle$一定是$\phi_n$的一个倍数,即
    \begin{equation*}
        \hat{B}|\phi_n \rangle = b_n |\phi_n \rangle
    \end{equation*}
    对于简并的情况,$\hat{B}|\phi_n$只能是所有本征值为$a_n$的本征态的线性组合。也就是说,如果
    \[ \hat{A}|\phi_{n+m} = a_n |\phi_{n+m},\ m=0,...,k\]
    并且$\hat{A},\hat{B}$对易,那么
    \[ \hat{B}|\phi_n \rangle = \sum_{m=0}^k c_m |\phi_{n+m} \rangle \]
    我们可以再将一个$\hat{B}$算符作用上来,得到
    \[ \hat{B}\hat{A} \sum_{m=0}^k c_m|\phi_{n+m}\rangle = \hat{A} \hat{B} \sum_{m=0}^k c_m|\phi_{n+m} \rangle = \hat{A} \sum_{m=0}^k c_m'|\phi_{n+m} \rangle \]
    在简并的情况下,可以通过构造得到$\hat{B}$的本征态。这是因为$\hat{B}$是一个Hermite算符,可以对角化:
    \[ \bm{U}^\dagger \bm{BU} = \bm{\Lambda} \]
    可以得到其本征态。

    我们已经讨论过动能算符和动量算符是对易的,如果
    \[ E_0 = \frac {\hat{p}^2}{2m} \]
    那么
    \[ p = \pm \sqrt{2mE_0} \]
    就是动能算符的两个本征态。也就是说,
    \[ \frac {p_0^2}{2m} |\psi\rangle = \frac {\hat{p}^2}{2m} (c_+|p_0\rangle + c_-|p_0\rangle) \]

    现在求解一维无限深势阱的能量本征态。其势能算符为
    \begin{equation*}
        V(x) = \left \{
            \begin{aligned}
                &0,\ x\in [-\frac L2, \frac L2 ]\\
                &\infty, \ \mathrm{otherwise}
            \end{aligned}
            \right.
    \end{equation*}
    Hamilton算符为
    \begin{equation*}
        \hat{H} = \frac {\hat{p}^2}{2m}
    \end{equation*}
    波函数只能在$[-\frac L2, \frac L2 ]$区间内,并且边界条件给出
    \begin{align*}
        \phi(x = -\frac L2) &= 0\\
        \phi(x = \frac L2) &= 0
    \end{align*}
    应有
    \begin{align*}
        \phi_n(x) &= c_+ \langle x|p_n\rangle + c_- \langle x|p_n\rangle\\
        &= \frac 1{\sqrt{2\pi \hbar}}(c_+\mathrm{e}^{\frac {\mathrm{i}xp_n}{\hbar}}+c_-\mathrm{e}^{-\frac {\mathrm{i}xp_n}{\hbar}})
    \end{align*}
    再加上边界条件
    \begin{align*}
        c_+\mathrm{e}^{\frac {\mathrm{i}Lp_n}{2\hbar}}+c_-\mathrm{e}^{-\frac {\mathrm{i}Lp_n}{2\hbar}} &= 0\\
        c_+\mathrm{e}^{-\frac {\mathrm{i}Lp_n}{2\hbar}}+c_-\mathrm{e}^{\frac {\mathrm{i}Lp_n}{2\hbar}} &= 0
    \end{align*}
    又有归一化条件
    \begin{align*}
        \langle \phi_n | \phi_n \rangle = \int_{-\frac L2}^{\frac L2} |\phi_n(x)|^2 \mathrm{d}x = 1
    \end{align*}
    定义算符$\hat{B}$满足
    \[ \langle x|\hat{B}| p \rangle = \langle x+\lambda |p\rangle \]
    由于
    \begin{align*}
        \langle x|\hat{B}| p \rangle = \langle x+\lambda |p\rangle = \frac {\mathrm{e}^{\frac {\mathrm{i}(x+\lambda)p}{\hbar}}}{\sqrt{2\pi\hbar}}
    \end{align*}
    显然地,应有
    \begin{equation*}
        \hat{B} = \mathrm{e}^{\frac {\mathrm{i}\lambda \hat{p}}{\hbar}}
    \end{equation*}
    这个算符称为\textbf{平移算符}。左矢形式表达为
    \[ \langle x| \mathrm{e}^{\frac {\mathrm{i}\lambda \hat{p}}{\hbar}} = \langle x+\lambda| \]
    右矢形式表达为
    \[ \mathrm{e}^{-\frac {\mathrm{i}\lambda \hat{p}}{\hbar}} | x\rangle = |x+\lambda \rangle \]
    我们使用平移算符,将一维势阱的体系作平移,将波函数平移到$[0,\frac L2]$的位置上。此时体系满足
    \begin{equation*}
        V(x) = \left \{
            \begin{aligned}
                &0,\ x\in [0, L]\\
                &\infty, \ \mathrm{otherwise}
            \end{aligned}
            \right.
    \end{equation*}
    由边界条件
    \begin{align*}
        \phi(0)= 0\\
        \phi(L) = 0
    \end{align*}
    得到
    \begin{align*}
        c_+ +c_- &= 0\\
        c_+\mathrm{e}^{\frac {\mathrm{i}Lp_n}{\hbar}}+c_-\mathrm{e}^{-\frac {\mathrm{i}Lp_n}{\hbar}} &= 0
    \end{align*}
    将前一个式子代入后一个,得到
    \[ c_+\mathrm{e}^{\frac {\mathrm{i}Lp_n}{\hbar}}-c_+ \mathrm{e}^{-\frac {\mathrm{i}Lp_n}{\hbar}} = 0 \]
    于是
    \begin{align*}
        2\mathrm{i}c_+ \sin{\frac {Lp_n}{\hbar}} = 0
    \end{align*}
    但是$c_+ \neq 0$(否则得到零解),所以
    \begin{align*}
        \frac {Lp_n}{\hbar} = n\pi
    \end{align*}
    即
    \[ p_n = \frac {n\pi \hbar}{L} \]
    $c_+$的选择取决于归一化条件,
    \[ \int_0^L |c|^2 \sin^2{\frac {n\pi x}L}\mathrm{d}x = 1 \]
    算出
    \[ c = \sqrt{\frac 2L} \]
    于是,一维无限深势阱的解为
    \[ \langle x|\phi_n \rangle = \sqrt{\frac 2L} \sin{\frac {n\pi x}L} \]
    本征值为
    \[ \epsilon_n = \frac {\hbar^2}{2m} \bigg(\frac {n\pi}L\bigg)^2 \]
    平移回来,得到
    \[ \langle x|\phi_n \rangle = \sqrt{\frac 2L} \sin{\bigg(\frac {n\pi}L\bigg(x+\frac L2\bigg)\bigg)}, \ n=1,2,3,... \]
    本征值和平移前一样。
    \begin{asg}
        第5次作业第3题(1):一维无限深势阱能量本征态在动量空间的表示。
    \end{asg}
    \begin{asg}
        第6次作业第1题(2):动量平移算符。
    \end{asg}

\section{20201120:一维势阱求解自由粒子问题}

    上一节在求解一维无限深势阱的过程中,引入了平移算符。我们想要了解$\mathrm{e}^{\frac {\mathrm{i}\lambda \hat{p}}{\hbar}} \hat{H} \mathrm{e}^{-\frac {\mathrm{i}\lambda \hat{p}}{\hbar}}$的性质。显然地,这是一个Hermite算符,并且新的算符和原来的Hamilton算符$\hat{H}$有相同的本征值。这是因为我们只改变了坐标的选取。
    \begin{asg}
        第6次作业第1题(2):证明这个结论。
    \end{asg}
    现在想要来模拟自由粒子,只需要让$L \to \infty$。对于自由粒子,如果给定温度$T$,动量应当满足Boltzmann分布:
    \[ \rho(p) = \sqrt{\frac {\beta}{2\pi m}}\mathrm{e}^{-\frac {\beta p^2}{2m}} \]
    计算能量的平均值
    \[ \langle \frac {p^2}{2m} \rangle = \frac 1{2\beta} \]

    如果
    \[ \hat{H} | \phi_n \rangle = \epsilon_n |\phi_n \]
    那么显然有
    \[ f(\hat{H})|\phi_n \rangle = f(\epsilon_n)|\phi_n \]
    定义Boltzmann算符$\mathrm{e}^{-\beta\hat{H}}$, 并且定义配分函数为
    \[ Z = \mathrm{Tr}\  \mathrm{e}^{-\beta \hat{H}} = \sum_n \langle n| \mathrm{e}^{-\beta \hat{H}} | n \rangle = \sum_n \mathrm{e}^{-\beta \epsilon_n }\]
    那么这个配分函数是否收敛呢?
    \[ Z = \sum_n \mathrm{e}^{-\beta \frac {\hbar^2}{2m} (\frac {n\pi}L)^2} \]
    定义$x_n = \frac nL$, 于是$\Delta x = \frac 1L$. 由此可以将求和变成积分。
    \begin{align*}
        Z &= L \sum_n \Delta x \mathrm{e}^{-\beta \frac {\hbar^2\pi^2 x^2}{2m} }\\
        &= L \int_0^{+\infty} \mathrm{e}^{-\beta \frac {\hbar^2\pi^2 x^2}{2m}} \mathrm{d}x\\
        &= L\sqrt{\frac m{2\pi \beta \hbar^2}}
    \end{align*}
    或者我们定义
    \[ \mathrm{Tr} \ \mathrm{e}^{-\frac {\beta \hat{p}^2}{2m}} = \int \mathrm{e}^{-\frac {\beta p^2}{2m}} \langle p|p\rangle \mathrm{d}p \]
    发现这里并不好处理,只能知道该值为$\infty$, 但不能给出具体的表达形式。这就是我们使用一维无限深势阱来近似自由粒子的原因。

    有了一维形式的配分函数,类比得到三维粒子为
    \[ Z = V \bigg(\frac m{2\pi \beta \hbar^2}\bigg)^{\frac 32} \]

    有了配分函数可以得到一维情况下能量的平均值:
    \begin{align*}
        \langle \hat{H} \rangle &= \frac {\mathrm{Tr} \ (\mathrm{e}^{-\beta \hat{H}} \hat{H})}Z
        = \frac {\sum_n \epsilon_n \mathrm{e}^{-\beta \epsilon_n}}{\sum_n \mathrm{e}^{-\beta \epsilon_n}}
        = -\frac {\partial}{\partial \beta} \ln{Z}
    \end{align*}
    将配分函数代入得到
    \[ \langle \hat{H} \rangle = \frac 1{2\beta} \]
    该结果和经典情况得到的结果是一致的。得到结论,自由粒子的体系经典和量子力学的结果是一致的。
    \begin{asg}
        第5次作业第3题(3):能量涨落的计算
    \end{asg}
    \begin{asg}
        第5次作业第3题(4):比热的计算
    \end{asg}
    \begin{asg}
        第5次作业第3题(5):用一维势阱求解共轭体系
    \end{asg}

\section{20201123:用一维势阱模型展开其他势能体系}

    我们解出了一维势阱能量本征态在位置空间的描述,现在要求在动量空间的描述:
    \[ \langle p|\phi_n \rangle = \int \langle p|x \rangle \langle x |\phi_n \rangle \mathrm{d}x \]

    两个有限维矩阵乘积的求迹:
    \begin{equation*}
        \mathrm{Tr} \ (\bm{AB}) = \mathrm{Tr} \ (\bm{BA})
    \end{equation*}
    证明是显然的,只需要直接展开
    \begin{align*}
        \mathrm{Tr} \ (\bm{AB}) &= \sum_i (\bm{AB})_{ii}
        = \sum_i \sum_k a_{ik}b_{ki}
        = \sum_k \sum_i b_{ki}a_{ik}
        = \sum_k \bm{(BA)}_{kk}
        = \mathrm{Tr} \ (\bm{BA})
    \end{align*}
    但对于无限维的,必须保证二重级数绝对收敛才能够交换次序才能成立。
    \begin{asg}
        第5次作业第3题(2):位置算符和动量算符乘积交换后迹是否相等?
    \end{asg}

    一维无限深势阱的能级差会随着$n$的增大而增大。
    \[ \Delta \epsilon = \frac {\hbar^2}{2m} \bigg(\frac {n\pi}L \bigg)^2 (2n+1) \]
    虽然如此,我们仍可以用一维无限深势阱的能量本征态来对其他的体系进行研究。比如一个势能算符为$\hat{V}'$时,势能矩阵元为
    \[
        \langle \phi_k | \hat{V}' | \phi_n \rangle = \int_{-\frac L2}^{+\frac L2} \phi_k^*(x) V(x) \phi_n(x) \mathrm{d}x 
    \]
    但是动能算符和原来是一样的。
    \[
        \langle \phi_k | \frac {\hat{p^2}}{2m} | \phi_n \rangle = \delta_{kn} \frac {\hbar^2}{2m} \bigg( \frac {\pi}L \bigg)^2 n^2
    \]
    由此可以得到Hamilton算符的矩阵元$\langle \phi_k | \hat{H} | \phi_n \rangle$,将它对角化就可以用来展开其他势能下的能量本征态。
    \begin{asg}
        第6次作业第2题:用一维无限深势阱展开一维谐振子的能量本征态和四次势的本征态。
    \end{asg}

    我们进入下一个话题:求解一维谐振子体系。一维谐振子的势能函数为
    \[ V(x) = \frac 12 m\omega^2x^2 \]
    Hamilton算符为
    \[ \hat{H} = \frac {\hat{p}^2}{2m} + \frac 12 m\omega^2 \hat{x}^2 \]
    类比复数域的
    \[ a^2 + b^2 = (a+b\mathrm{i})(a-b\mathrm{i}) \]
    对于数字这样分解是可以的,但是对于算符来说,只有对易的算符才成立。先考虑数字的情况
    \[ \frac {p^2}{2m} + \frac 12 m\omega^2 x^2 = \frac 12 (\frac p{\sqrt{m}} + \mathrm{i}\sqrt{m}\omega x)(\frac p{\sqrt{m}} - \mathrm{i}\sqrt{m}\omega x) \]
    可以先把能量$\hbar \omega$提出来,这样就可以操作里面的没有量纲的算符,会更方便一些。

\section{20201127:一维谐振子的求解(1)}

    继续讨论一维谐振子的求解问题。定义算符
    \[ \hat{c} = \frac {\hat{p}}{\sqrt{2m}} + \mathrm{i}\sqrt{\frac m2}\omega \hat{x} \]
    算符的对易满足如下性质:
    \begin{align*}
        [\hat{A}+\hat{B},\hat{C}] &= [\hat{A},\hat{C}]+[\hat{B},\hat{C}]\\
        [\alpha \hat{A}, \beta \hat{B}] &= \alpha \beta [\hat{A},\hat{B}]\\
        [\hat{A}\hat{B},\hat{C}] &= \hat{A}[\hat{B},\hat{C}] + [\hat{A},\hat{C}]\hat{B}
    \end{align*}
    \begin{asg}
        第6次作业第3题(1):证明上述结论
    \end{asg}
    可以计算
    \begin{align*}
        [\hat{c},\hat{c}^\dagger] &= [d_1\hat{p} +\mathrm{i}d_2\hat{x}, d_1\hat{p} - \mathrm{i}d_2\hat{x}]\\
        &= [d_1\hat{p}, d_1\hat{p} - \mathrm{i}d_2\hat{x}] + [\mathrm{i}d_2\hat{x}, d_1\hat{p} - \mathrm{i}d_2\hat{x}]\\
        &= [d_1\hat{p}, -\mathrm{i}d_2\hat{x}] + [\mathrm{i}d_2\hat{x}, d_1\hat{p}]\\
        &= -\mathrm{i}d_1d_2[\hat{p},\hat{x}] + \mathrm{i}d_1d_2[\hat{x},\hat{p}]\\
        &= -2d_1d_2\hbar\\
        &= -\hbar \omega
    \end{align*}
    为了让算符无量纲化,定义
    \begin{align*}
        \hat{a} = \frac {\hat{c}}{\sqrt{\hbar \omega}} =\frac 1{\sqrt{2}} \bigg(\frac {\hat{p}}{\sqrt{m\hbar\omega}} + \mathrm{i}\sqrt{\frac {m\omega}{\hbar}} \hat{x}\bigg)
    \end{align*}
    \begin{asg}
        第6次作业第3题(2):证明$\hat{a}\hat{a}^\dagger$和$\hat{a}^\dagger\hat{a}$都是Hermite算符。
    \end{asg}
    显然
    \[ [\hat{a},\hat{a}^\dagger] = \hat{a}\hat{a}^\dagger - \hat{a}^\dagger \hat{a} =  -1 \]
    可以用$\hat{a}$写出Hamilton算符:
    \[ \hat{H} = \frac {\hbar \omega}2 (\hat{a}\hat{a}^\dagger + \hat{a}^\dagger \hat{a}) \]
    可以计算
    \begin{align*}
        [\hat{a}^\dagger\hat{a}, \hat{a}\hat{a}^\dagger] &= \hat{a}^\dagger [\hat{a},\hat{a}\hat{a}^\dagger] +  [\hat{a}^\dagger,\hat{a}\hat{a}^\dagger]\hat{a}\\
        &= \hat{a}^\dagger\hat{a}[\hat{a},\hat{a}^\dagger] + [\hat{a}^\dagger,\hat{a}]\hat{a}^\dagger \hat{a}\\
        &= 0
    \end{align*}
    \begin{asg}
        第6次作业第3题(3):证明
        \[ \hat{H} = \frac {\hbar \omega}2 (\hat{a}\hat{a}^\dagger + \hat{a}^\dagger \hat{a}) \]
    \end{asg}
    上面结果也给出了
    \[ \hat{a}^\dagger\hat{a} = \hat{a}\hat{a}^\dagger + 1 \]
    于是
    \[ \hat{H} = \hbar \omega\bigg(\hat{a}\hat{a}^\dagger + \frac 12\bigg)\]
    现在定义
    \[ \hat{b} = \frac 1{\sqrt{2}}\bigg(\sqrt{\frac {m\omega}{\hbar}}\hat{x} + \frac {\mathrm{i}\hat{p}}{\sqrt{m\omega\hbar}}\bigg) \]
    用相同的方法得到 
    \[ \hat{H} = \hbar \omega\bigg(\hat{b}^\dagger\hat{b}+ \frac 12\bigg) \]
    并且
    \[ [\hat{b}, \hat{b}^\dagger] = 1 \]
    定义
    \[ \hat{N} = \hat{b}^\dagger\hat{b} \]
    于是
    \[ \hat{H} = \hbar \omega \bigg(\hat{N}+\frac 12\bigg) \]
    因此 
    \[ [\hat{N},\hat{H}] = 0 \]
    两个对易的算符有相同的本征态。假设
    \[ \hat{N}|\phi_n \rangle = \lambda_n |\phi_n \rangle \]
    则
    \begin{align*}
        \hat{H}|\phi_n \rangle = \bigg(\hat{N}+\frac 12\bigg)\hbar\omega|\phi_n \rangle = \bigg(\lambda_n + \frac 12\bigg)\hbar\omega|\phi_n\rangle
    \end{align*}
    并且
    \[ \langle \phi_n|\hat{N}|\phi_n \rangle = \lambda_n \langle \phi_n |\phi_n \rangle = \lambda_n \]
    而
    \[ \langle \phi_n|\hat{N}|\phi_n \rangle =  \langle \phi_n |\hat{b}^\dagger\hat{b}|\phi_n \rangle = \lambda_n \]
    令
    \[ |\psi_n \rangle = \hat{b}|\phi_n\rangle \]
    那么
    \[ \langle \phi_n|\hat{N}|\phi_n \rangle =  \langle \phi_n |\hat{b}^\dagger\hat{b}|\phi_n \rangle = \langle \psi_n|\psi_n \rangle = \lambda_n \geqslant 0 \]
    这样证明了$\hat{N}$的本征值必然是非负数。

    那么$|\psi_n\rangle$是否仍然是$\hat{N}$的本征态呢?计算
    \begin{align*}
        \hat{N}|\psi_n\rangle = \hat{b}^\dagger\hat{b}^2|\phi_n\rangle = (\hat{b}\hat{b}^\dagger\hat{b}- \hat{b})|\phi_n\rangle = \hat{b}(\hat{N}-\hat{I})|\phi_n\rangle = (\lambda_n-1)\hat{b}|\phi_n\rangle = (\lambda_n - 1)|\psi_n\rangle
    \end{align*}
    这说明$\hat{b}$作用于$\hat{N}$的本征态以后得到的态仍然是$\hat{N}$的本征态,且本征值减少1. 于是可以设
    \[ \hat{b}|\phi_n\rangle = |\psi_n\rangle = \sqrt{\lambda_n}|\phi_m\rangle \]
    将$\hat{N}$作用上来,得到
    \[ \hat{N}\sqrt{\lambda_n}|\phi_m \rangle = \sqrt{\lambda_n}(\lambda_n-1)|\phi_m\rangle \]
    这构造了一个循环,将$\hat{b}$作用在$\hat{N}$的本征态上,得到一个新的$\hat{N}$的本征态,且$\hat{N}$的本征值减少1,并且它仍然是非负的。依次类推,总会有一个态$\hat{N}$的本征值为0,且$\hat{N}$的本征值都是整数。如果$\hat{N}$的本征值为0,此时如果再用$\hat{b}$作用,则得到零向量。所以
    \[ \hat{N}|\phi_n \rangle = n|\phi_n\rangle \]
    故将$\hat{N}$称为\textbf{数值算符}。

\section{20201130:一维谐振子的求解(2)}

    一维谐振子在通过$\hat{b}$算符的作用时,$\hat{N}$的本征值下降1,故吧$\hat{b}$称为\textbf{下降算符}。最终本征值下降到0时,应有
    \[ \hat{b}|\phi_0 \rangle = 0 \]
    可以推出
    \[ \hat{N}|\phi_0 \rangle = 0 \]
    求解
    \[ \langle x|\hat{b}|\phi_0 \rangle = 0\]
    将$\hat{b}$的定义代入,得到
    \begin{align*}
        \langle x|\sqrt{\frac 12}\bigg(\sqrt{\frac {m\omega}{\hbar}}\hat{x}+ \frac {\mathrm{i}\hat{p}}{\sqrt{m\omega\hbar}}\bigg)|\phi_0 \rangle = 0
    \end{align*}
    解得
    \[ \langle x|\phi_0\rangle = \bigg(\frac {m\omega}{\pi\hbar} \bigg)^{\frac 14} \mathrm{e}^{-\frac {m\omega}{2\hbar}x^2} \]
    求出基态的能量
    \[ \hat{H}|\phi_0 \rangle = \frac 12 \hbar \omega |\phi_0 \rangle \]
    基态也有一定的能量,称为\textbf{零点能}。

    现在研究一下$\hat{b}^\dagger$作用于$\hat{N}$的本征态上。
    \begin{align*}
        \hat{N}\hat{b}^\dagger |\phi_n \rangle = (\hat{b}^\dagger \hat{b})\hat{b}^\dagger |\phi_n \rangle = \hat{b}^\dagger (\hat{b}^\dagger\hat{b}+\hat{I})|\phi_n \rangle = (\lambda_n+1) \hat{b}^\dagger |\phi_n \rangle
    \end{align*}
    所以,$\hat{b}^\dagger$作用在$\hat{N}$的本征态上还会得到$\hat{N}$的本征态,会使得$\hat{N}$的本征值上升1,于是将$\hat{b}^\dagger$称为\textbf{上升算符}。

    同理可以得到
    \[ \hat{b}^\dagger|\phi_n\rangle = \sqrt{\lambda_n+1}|\phi_m\rangle \]
    如果想要得到各个激发态的波函数,可以通过用$\hat{b}^\dagger$不断作用在基态的波函数上面:
    \[ |\phi_n \rangle = \frac 1{\sqrt{n!}} \hat{b}^{\dagger n} | \phi_0 \rangle \]

    \begin{asg}
        第6次作业第4题:求出一维谐振子的第$n$个能级的波函数及其势能。
    \end{asg}

\section{20201204:时间演化算符}

    前面我们用一维无限深势阱来展开不同的势能函数,有了一维谐振子的解我们也可以用一维谐振子的解来展开其他势能函数的情况。

    有了一维谐振子的解,我们可以拓展到多维,类比之前的简谐振动分析,得到能量的本征值为
    \[ E = \sum_{j=1}^F \bigg(n_j+\frac 12\bigg)\hbar \omega_j \]
    在简正坐标下的波函数为
    \[ \langle \bm{Q}|\psi \rangle = \prod_{j=1}^F \langle Q_j | \phi_{n_j} \rangle \]
    其中
    \[ \langle Q_j|\phi_{n_j} \rangle = \bigg(\frac {\omega}{\pi \hbar}\bigg)^{\frac 14} \mathrm{e}^{-\frac {\omega_j}{2\hbar} Q_j^2} \]
    注意此处没有质量,因为它被概率在简正坐标变换时引入的Jacobi行列式约掉了。

    如果Hamilton函数为两个Hamilton函数之和
    \[ \hat{H} = \hat{H}_1 + \hat{H}_2 \]
    设它们本征态为$\phi_{n_1}^{(1)}$和$\phi_{n_2}^{(2)}$
    于是
    \[\hat{H} |\phi_{n_1}^{(1)} \rangle |\phi_{n_2}^{(2)} = \epsilon_{n_1}|\phi_{n_1}^{(1)} \rangle \otimes |\phi_{n_2}^{(2)} + \epsilon_{n_2}|\phi_{n_2}^{(2)} \rangle \otimes |\phi_{n_1}^{(1)} = (\epsilon_{n_1}+\epsilon_{n_2})|\phi_{n_1}^{(1)} \rangle \otimes |\phi_{n_2}^{(2)}\]
    更普遍地,对于多维谐振子,应有
    \[ \hat{H} = \sum_{j=1}^F \hat{H}_j \]
    其中 
    \[ \hat{H}_j = \frac 12 \hat{P}_j^2 + \frac 12 \omega_j^2 Q_j^2 \]

    如果我们已知了
    \[ \hat{H}|\phi_n\rangle = \epsilon_n|\phi_n\rangle \]
    那么
    \[ \mathrm{e}^{-\frac {\mathrm{i}\hat{H}t}{\hbar}}|\phi_n\rangle = \mathrm{e}^{-\frac {\mathrm{i}\epsilon_n t}{\hbar}}|\phi_n\rangle \]
    含时的Schrodinger方程为
    \[ \mathrm{i}\hbar \frac {\partial}{\partial t}|\psi(t)\rangle = \hat{H} |\psi(t)\rangle \]
    注意时间在量子力学中并没有算符,而是一个参量。由含时的Schrodinger方程可以推出
    \[ \mathrm{e}^{-\frac {\mathrm{i}\hat{H}t}{\hbar}}|\psi(0)\rangle = |\psi(t) \rangle \]
    所以把$\mathrm{e}^{-\frac {\mathrm{i}\hat{H}t}{\hbar}}$称为\textbf{时间演化算符}。可以得到任意物理量在时间$t$的平均值 
    \[ \langle \hat{B}(t) \rangle = \langle \psi(t)|\hat{B} | \psi(t) \rangle = \langle \psi(0) |\mathrm{e}^{\frac {\mathrm{i}\hat{H}t}{\hbar}} \hat{B} \mathrm{e}^{-\frac {\mathrm{i}\hat{H}t}{\hbar}}|\psi(0) \rangle \]
    定义$\mathrm{e}^{\frac {\mathrm{i}\hat{H}t}{\hbar}} \hat{B} \mathrm{e}^{-\frac {\mathrm{i}\hat{H}t}{\hbar}}$为算符$\hat{B}$的\textbf{Heisenberg算符}.

    时间演化算符可以写在能量本征态上
    \[ \mathrm{e}^{-\frac {\mathrm{i}\hat{H}t}{\hbar}} = \sum_n \mathrm{e}^{-\frac {\mathrm{i}\epsilon_n t}{\hbar}} |\phi_n \rangle \langle\phi_n| \]
    代入,得到 
    \begin{align*}
        \langle \hat{B}(t) \rangle &= \langle \psi(t)|\hat{B} | \psi(t) \rangle\\
        &= \langle \psi(0) |\mathrm{e}^{\frac {\mathrm{i}\hat{H}t}{\hbar}} \hat{B} \mathrm{e}^{-\frac {\mathrm{i}\hat{H}t}{\hbar}}|\psi(0) \rangle\\
        &= \langle \psi(0) |\sum_n \mathrm{e}^{\frac {\mathrm{i}\epsilon_n t}{\hbar}} |\phi_n \rangle \langle\phi_n|\hat{B}|\sum_m \mathrm{e}^{-\frac {\mathrm{i}\epsilon_m t}{\hbar}} |\phi_m \rangle \langle\phi_m|\psi(0)\rangle\\
        &= \sum_{m,n} \langle \psi(0)|\phi_n\rangle \langle \phi_n|\hat{B}|\phi_m \rangle \langle \phi_m|\psi(0)\rangle \mathrm{e}^{\frac {\mathrm{i}(\epsilon_n-\epsilon_m)t}{\hbar}}
    \end{align*}
    \begin{asg}
        第7次作业第1题:高维谐振子$t$时刻的物理量
    \end{asg}

\section{20201207:一维谐振子经典和量子处理的比较}

    \begin{asg}
        第7次作业第2题:高维谐振子的时间自关联函数计算
    \end{asg}

    光谱的定义为
    \[ I(\omega) = \int_{-\infty}^{+\infty} \langle \hat{A}(0)\hat{A}(t)\rangle \mathrm{e}^{\frac {\mathrm{i}\omega t}{\hbar}} \mathrm{d}t \]
    如果$\hat{A} = \hat{M}$则为红外光谱;如果$\hat{A} = \hat{\beta}$则为Raman光谱。

    我们比较一下谐振子的经典和量子描述。经典配分函数为
    \[ Z_\mathrm{cl} = \int \mathrm{e}^{-\beta H(x,p)}\frac {\mathrm{d}x\mathrm{d}p}{2\pi\hbar} = \frac 1{\beta\hbar\omega} \]
    量子体系的配分函数为
    \[ Z_\mathrm{Q} = \mathrm{Tr} \ \mathrm{e}^{-\beta \hat{H}} = \sum_n \mathrm{e}^{-\beta\epsilon_n} = \frac 1{2\sinh{\frac {\beta\hbar\omega}2}} \]
    这两个结果在$\beta\hbar\omega \to 0$时是一致的。如果
    $\beta \to 0$则对应高温极限;$\hbar \to 0$对应经典极限;$\omega \to 0$对应能级差很小,也逼近经典情况。
    根据$m\omega^2 = k$, 增大约化质量会使得$\omega$减小,这就是同位素效应。

    有了配分函数就可以求出各个热力学函数。定义
    \[ u = \beta\hbar\omega \]
    自由能为
    \begin{align*}
        F_\mathrm{cl} &= -\frac 1{\beta} \ln{Z_\mathrm{cl}} = \frac 1{\beta} \ln{u}\\
        F_\mathrm{Q} &= -\frac 1{\beta} \ln{Z_\mathrm{Q}} = \frac 1{\beta}\ln{\sinh{\frac u2}}+ \frac {\ln{2}}{\beta}
    \end{align*}
    熵为
    \begin{align*}
        S_\mathrm{cl} &= -\bigg(\frac {\partial F_{\mathrm{cl}}}{\partial T}\bigg)_V = -k_B \ln{u} + k_B\\
        S_\mathrm{Q} &= -\bigg(\frac {\partial F_{\mathrm{Q}}}{\partial T}\bigg)_V = -k_B \ln{\sinh{\frac u2}} + \frac {k_Bu}2 \coth{\frac u2} - k_B \ln{2}
    \end{align*}
    内能为
    \begin{align*}
        U_\mathrm{cl} &= -\frac {\partial}{\partial \beta}\ln{Z_\mathrm{cl}} = \frac 1{\beta}\\
        U_\mathrm{Q} &= -\frac {\partial}{\partial \beta}\ln{Z_\mathrm{Q}} = \frac {u}{2\beta} \coth{\frac u2}
    \end{align*}
    热容为
    \begin{align*}
        C_{V\mathrm{cl}} = -k_B \beta^2 \bigg(\frac {\partial U_\mathrm{cl}}{\partial T}) &= k_B\\
        C_{V\mathrm{Q}} = -k_B \beta^2 \bigg(\frac {\partial U_\mathrm{Q}}{\partial T}) &= k_B \frac {(\frac u2)^2}{\sinh^2{\frac u2}}
    \end{align*}
    定义\textbf{量子校正因子}
    \[ Q(\frac u2) = \frac u2 \coth{\frac u2} \]
    于是 
    \[ \langle \hat{H} \rangle = U = \frac {Q(\frac u2)}{\beta} \]
    当$u \to 0$, $Q(\frac u2) \to 1$,接近经典结果;当$u \to +\infty$,$\frac {Q(\frac u2)}{\frac u2} \to 1$, 于是
    \[U \to \frac {\hbar \omega}2 \]
    能量即为零点能,即都聚集在基态。

    对于热容,如果$u \to +\infty$则有$C_V \to 0$

\section{20201221:传播子的计算}

    如果求出传播子
    \begin{equation*}
        \langle x_0 | \mathrm{e}^{-\frac {\mathrm{i}\hat{H}t}{\hbar}} | y_0 \rangle
    \end{equation*}
    那么就可以求解含时Schrodinger方程,这时就将研究对象从波函数变为传播子。所以现在需要求解传播子。

    首先研究$\mathrm{e}^{\lambda \hat{A}}\mathrm{e}^{\lambda \hat{B}}$和$\mathrm{e}^{\lambda (\hat{A}+\hat{B})}$的关系。应有
    \begin{equation*}
        \mathrm{e}^{\lambda \hat{A}} \mathrm{e}^{\lambda \hat{B}} = \mathrm{e}^{\lambda (\hat{A}+\hat{B}) + \frac 12 \lambda^2 [\hat{A},\hat{B}] + O(\lambda^2)}
    \end{equation*}
    如果$\lambda \to 0$, 可以忽略二阶无穷小量,则有
    \begin{equation*}
        \mathrm{e}^{\lambda \hat{A}} \mathrm{e}^{\lambda \hat{B}} = \mathrm{e}^{\lambda (\hat{A}+\hat{B})}
    \end{equation*}

    将时间平均分为$N$份,令$\lambda = \frac tN$, 并令$N \to \infty$.所以
    \begin{equation*}
        \mathrm{e}^{\frac tN (-\frac {\mathrm{i}}{\hbar}) \hat{K}} \mathrm{e}^{\frac tN (-\frac {\mathrm{i}}{\hbar}) \hat{V}} =\mathrm{e}^{\frac tN (-\frac {\mathrm{i}}{\hbar}) \hat{H}} 
    \end{equation*}
    代入传播子,得到
    \begin{align*}
        \langle x_0 | \mathrm{e}^{-\frac {\mathrm{i}\hat{H}t}{N\hbar}} | y_0 \rangle &= \langle x | \mathrm{e}^{\frac tN (-\frac {\mathrm{i}}{\hbar}) \frac {\hat{p}^2}{2m}}
        \mathrm{e}^{\frac tN (-\frac {\mathrm{i}}{\hbar}) \hat{V}} |y \rangle\\
        &= \langle x | \mathrm{e}^{\frac tN (-\frac {\mathrm{i}}{\hbar}) \frac {\hat{p}^2}{2m}} |y \rangle \mathrm{e}^{\frac tN (-\frac {\mathrm{i}}{\hbar}) V(y)}\\
        &= \int \langle x | \mathrm{e}^{\frac tN (-\frac {\mathrm{i}}{\hbar}) \frac {\hat{p}^2}{2m}} |p \rangle \langle p |y \rangle \mathrm{d}p \times \mathrm{e}^{\frac tN (-\frac {\mathrm{i}}{\hbar}) V(y)}\\
        &= \int \langle x|p \rangle \langle p|y \rangle \mathrm{e}^{\frac tN (-\frac {\mathrm{i}}{\hbar}) \frac {p^2}{2m}} \mathrm{d}p \times \mathrm{e}^{\frac tN (-\frac {\mathrm{i}}{\hbar}) V(y)}\\
        &= \frac 1{2\pi \hbar} \int \mathrm{e}^{\frac {i(x-y)p}{\hbar}} \mathrm{e}^{\frac tN (-\frac {\mathrm{i}}{\hbar}) \frac {p^2}{2m}} \mathrm{d}p \times \mathrm{e}^{\frac tN (-\frac {\mathrm{i}}{\hbar}) V(y)}
    \end{align*}
    根据Gauss积分
    \begin{equation*}
        \int_{-\infty}^{+\infty} \mathrm{e}^{-ax^2+bx} \mathrm{d}x = \sqrt{\frac {\pi}a} \mathrm{e}^{-\frac {b^2}{4a}}
    \end{equation*}
    由此得到传播子为
    \begin{align*}
        \langle x_0 | \mathrm{e}^{-\frac {\mathrm{i}\hat{H}t}{N\hbar}} | y_0 \rangle &= \sqrt{\frac {mN}{2\pi\mathrm{i} \hbar t}} \mathrm{e}^{-\mathrm{i}\frac {mN(x-y)^2}{2\hbar t}}\mathrm{e}^{-\frac {\mathrm{i}t}{N\hbar} V(y)}
    \end{align*}
    前两项来自于动能算符,第三项来自于势能算符。动能算符和势能算符虽然不对易,但是在$N \to \infty$时可以得到这个结果。

    自由粒子体系的势能为0,所以可以不需要把时间分成$N$份,而是直接对整个传播子来计算。把$V=0,N=1$代入上式,即得到
    \begin{align*}
        \langle x_0 | \mathrm{e}^{-\frac {\mathrm{i}\hat{H}t}{\hbar}} |y_0 \rangle &= \sqrt{\frac {m}{2\pi\mathrm{i} \hbar t}} \mathrm{e}^{-\mathrm{i}\frac {m(x-y)^2}{2\hbar t}}
    \end{align*}
    如果推广到$F$维体系,则有
    \begin{align*}
        \langle \bm{x_0} | \mathrm{e}^{-\frac {\mathrm{i}\hat{H}t}{\hbar}} |\bm{y_0} \rangle &= (\frac {1}{2\pi\mathrm{i} \hbar t})^{\frac F2} |\bm{M}|^{\frac 12} \mathrm{e}^{-\mathrm{i}\frac {\bm{(x-y)}^{\mathrm{T}} \bm{M (x-y)}}{2\hbar t}}
    \end{align*}
    可以将传播子写成
    \begin{align*}
        \langle y_0 | \mathrm{e}^{-\frac {\mathrm{i}\hat{H}t}{\hbar}} |x_0 \rangle &= C(t) \mathrm{e}^{\frac {\mathrm{i}S(x(t))}{\hbar}}
    \end{align*}
    在经典情况下写出作用量
    \begin{align*}
        S(x(t)) = \int_0^t \mathcal{L}(x,\dot{x},t') \mathrm{d}t' = \frac 12 \int_0^t m\dot{x}^2 \mathrm{d}t'
    \end{align*}
    Lagrange函数会满足Euler-Lagrange方程,而对于自由粒子,Lagrange函数不显含坐标,所以
    \begin{align*}
        \frac {\mathrm{d}}{\mathrm{d}t} (m \dot x) = 0
    \end{align*}
    由此可见,速度不随时间变化,且
    \begin{align*}
        \dot{x} = \frac {y_0-x_0}t
    \end{align*}
    所以,上述作用量积分的结果为
    \begin{align*}
        S = \frac {m(y_0 - x_0)^2}{2t}
    \end{align*}
    显然地,这个结果代入上面写出的传播子表达式相吻合。现在希望能把$C(t)$求出。给定初始条件
    \begin{equation*}
        t \to 0,~~~~~\langle y_0|x_0\rangle = \delta(y_0-x_0)
    \end{equation*}
    计算出
    \begin{align*}
        \int_{-\infty}^{+\infty} \mathrm{e}^{\frac {\mathrm{i}}{\hbar} \frac {m(y_0 - x_0)^2}{2t}}\mathrm{d}y_0 = C(t) \sqrt{\frac {2\pi\mathrm{i}\hbar t}m}
    \end{align*}
    于是 
    \begin{align*}
        C(t) = \sqrt{\frac m{2\pi\mathrm{i}\hbar t}} D(t)
    \end{align*}
    其中$D(0) = 1$. 现在希望证明$D(t) = 1$.

    计算
    \begin{align*}
        -\mathrm{i}\hbar \frac {\partial}{\partial t} \langle y_0 | \mathrm{e}^{-\frac {\mathrm{i}\hat{H}t}{\hbar}} |x_0 \rangle =  \langle y_0 | \hat{H} \mathrm{e}^{-\frac {\mathrm{i}\hat{H}t}{\hbar}} |x_0 \rangle
        = -\frac {\hbar^2}{2m} \frac {\partial^2}{\partial y_0^2} \langle y_0 | \mathrm{e}^{-\frac {\mathrm{i}\hat{H}t}{\hbar}} |x_0 \rangle
    \end{align*}

    \begin{asg}
        第8次作业第四题
    \end{asg}

    如果不是自由体系,则使用\textbf{多边折线方案}。

\section{20201225:量子力学和经典力学在路径积分表象下的同构}

    假设已经得到传播子
    \begin{equation*}
        \langle x | \mathrm{e}^{-\frac {i \hat{H}t}{\hbar}} |y \rangle = \sqrt{\frac m{2\pi \mathrm{i}\hbar t}} \mathrm{e}^{\mathrm{i}\frac {m(x-y)^2}{2t\hbar}}
    \end{equation*}
    路径积分是在空间中连接所有$x,y$的路径都要进行考虑。所以,传播子是对所有路径求和
    \begin{align*}
        \langle x | \mathrm{e}^{-\frac {i \hat{H}t}{\hbar}} = \sum_{\mathrm{all~paths}} C_t \mathrm{e}^{\mathrm{i}S_t\hbar}
    \end{align*}
    其中 
    \begin{align*}
        C_t &= \sqrt{\frac m{2\pi \mathrm{i}\hbar t}} \\
        S_t &= \int_0^t \mathcal{L}(x,\dot{x},t') \mathrm{d}t' = \int_0^t (\frac 12 m \dot{x}^2 - V(x)) \mathrm{d}t'
    \end{align*}

    量子体系下的Boltzmann分布为
    \begin{equation*}
        \mathrm{e}^{-\beta \hat{H}} = \sum_n \mathrm{e}^{-\beta E_n} |\phi_n \rangle \langle \phi_n|
    \end{equation*}
    利用了Schodinger方程
    \begin{equation*}
        \hat{H} |\phi_n \rangle = E_n |\phi_n \rangle
    \end{equation*}
    类似地,可以作和
    \begin{align*}
        \mathrm{e}^{-\frac {\mathrm{i}\hat{H}t}{\hbar}} = \sum_n \mathrm{e}^{-\frac {\mathrm{i}E_n t}{\hbar}} |\phi_n \rangle \langle \phi_n|
    \end{align*}
    上述两个式子可以对应起来,区别在于第二个方程中的时间是虚数,称为\textbf{虚时间}。对应关系为
    \begin{equation*}
        t = -\mathrm{i}\hbar \beta
    \end{equation*}
    显然地,高温对应虚时间的短时,低温对应虚时间的长时。同样可求出虚时间下的传播子
    \begin{align*}
        \langle x|\mathrm{e}^{-\beta \hat{H}}|y\rangle
    \end{align*}
    求出配分函数
    \begin{align*}
        Z &= \mathrm{Tr}(\mathrm{e}^{-\beta \hat{H}}) \\
        &= \sum_n \langle n| \mathrm{e}^{-\beta \hat{H}}|n\rangle \\
        &= \int \sum_n \langle n| \mathrm{e}^{-\beta \hat{H}}|x \rangle \langle x|n\rangle \mathrm{d}x\\
        &= \int \langle x|\sum_n|n \rangle \langle n| \mathrm{e}^{-\beta \hat{H}}|x\rangle \mathrm{d}x\\
        &= \int \langle x|\mathrm{e}^{-\beta \hat{H}}|x\rangle \mathrm{d}x
    \end{align*}
    在路径积分的语言下,可以放弃态的概念,也不需要有波函数,只要有传播子,就有配分函数,也就有了所有的热力学函数。要求这个积分中的传播子,将$\beta$分为$N$份.有
    \begin{align*}
        \langle x|\mathrm{e}^{-\beta \hat{H}}|x\rangle  = \int \langle x_0|\mathrm{e}^{-\frac {\beta \hat{H}}N} |x_1 \rangle ... \langle x_{n-2}|\mathrm{e}^{-\frac {\beta \hat{H}}N} |x_{n-1} \rangle \langle x_{n-1}|\mathrm{e}^{-\frac {\beta \hat{H}}N} |x_N \rangle \prod_{i=1}^{N-1}\mathrm{d}x_i
    \end{align*}
    其中$x_0=x_N = x$. 如果$N \to \infty$, 则可以把动能项和势能项分开。用和之前传播子计算类似的方法得到
    \begin{align*}
        \langle x_j|\mathrm{e}^{-\Delta \beta \hat{H}} |x_{j+1} \rangle &= \sqrt{\frac m{2\pi \hbar^2 \Delta \beta}} \mathrm{e}^{-\frac {m(x_j- x_{j+1})^2}{2\hbar^2 \Delta \beta}} \mathrm{e}^{-\Delta \beta V(x_{j+1})}
    \end{align*}
    定义$\omega_N^2 = \frac N{\hbar^2 \Delta \beta^2} = \frac N{\hbar^2 \beta^2}$,则有
    \begin{align*}
        \langle x_j|\mathrm{e}^{-\Delta \beta \hat{H}} |x_{j+1} \rangle &= \sqrt{\frac m{2\pi \hbar^2 \Delta \beta}} \mathrm{e}^{-\frac {\beta m\omega_N^2(x_j- x_{j+1})^2}{2}} \mathrm{e}^{-\Delta \beta V(x_{j+1})}
    \end{align*}
    代入得到 
    \begin{align*}
        \langle x|\mathrm{e}^{-\beta \hat{H}}|x\rangle  = \bigg(\frac {mN}{2\pi \hbar^2 \beta}\bigg)^{\frac N2} \int \mathrm{e}^{-\sum_{j=0}^{N-1} \frac {\beta}2 m \omega_N^2 (x_{j+1} - x_j)^2} \mathrm{e}^{-\Delta \beta \sum_{j=0}^{N-1}V(x_j)}\prod_{i=1}^{N-1}\mathrm{d}x_i
    \end{align*}
    这可以看作$N$个点组成的环两两用弹簧连接,且每个点都额外受外力作用。它可以写成
    \begin{align*}
        \langle x|\mathrm{e}^{-\beta \hat{H}}|x\rangle  = \int C(N) \mathrm{e}^{-\beta V_{\mathrm{eff}}(\bm{x})} \mathrm{d}\bm{x}
    \end{align*}
    积分不太容易做,可以在这里插入一个关于“动量”的积分
    \begin{align*}
        \langle x|\mathrm{e}^{-\beta \hat{H}}|x\rangle  &= \int D(N) \mathrm{e}^{-\beta V_{\mathrm{eff}}(\bm{x})} \mathrm{d}\bm{x} \int \mathrm{d}\bm{p} \mathrm{e}^{-\frac {\beta}2 \bm{p}^{\mathrm{T}}\bm{M}^{-1} \bm{p}}\\
        &= \int D(N) \mathrm{e}^{-\beta H_{\mathrm{eff}}(\bm{x,p})} \mathrm{d}\bm{x} \mathrm{d}\bm{p}
    \end{align*}
    如果$N=1$,那么就是经典统计力学的结果。这体现了量子力学和经典统计力学的同构。

    如果关心含时Schodinger方程:
    \begin{equation*}
        \mathrm{i}\hbar \frac {\partial}{\partial t} | \psi(t) \rangle = \hat{H}|\psi(t) \rangle
    \end{equation*}
    将这个改成对于$\beta$的方程:
    \begin{align*}
        -\frac {\partial}{\partial \beta} |\psi(\beta) \rangle = \hat{H} |\psi(\beta) \rangle
    \end{align*}

    任意给定一个初态,可以写成Hamilton函数的本征函数的线性组合
    \begin{align*}
        |\psi(0) \rangle = \sum_n c_n |\phi_n\rangle
    \end{align*}
    并且
    \begin{align*}
        |\psi(\beta) \rangle &= \mathrm{e}^{-\beta \hat{H}} |\phi(0) \rangle = \sum_n c_n\mathrm{e}^{-\beta E_n}|\phi_n \rangle = \mathrm{e}^{-\beta E_0} \sum_n c_n \mathrm{e}^{-\beta(E_n-E_0)} |\phi_n \rangle
    \end{align*}
    因此,当$\beta \to \infty$时,得到的就是基态。基于此开发出了\textbf{量子Monte-Carlo算法}

    如果要求第一激发态,只需求出系数,把基态从初始条件中减去,得到的新的最低的能量对应的态就是第一激发态。

\section{20201228:连续性方程}

    考虑一个一维问题:
    \begin{equation*}
        \mathrm{i}\hbar \frac {\partial}{\partial t} \psi(x,t) = \bigg(-\frac {\hbar^2}{2m} \frac {\partial ^2}{\partial x^2} + V(x)\bigg)\psi(x,t)
    \end{equation*}
    波函数满足
    \begin{equation*}
        \int_{-\infty}^{+\infty} |\psi(x,t)|^2 \mathrm{d}x = 1
    \end{equation*}
    这可以类比守恒量。比如电荷守恒
    \begin{equation*}
        Q = \int \rho(\bm{x},t) \mathrm{d}\bm{x}
    \end{equation*}
    或者电流密度
    \begin{equation*}
        j(\bm{x},t) = \rho(\bm{x},t)v(\bm{x},t)
    \end{equation*}
    满足\textbf{连续性方程}
    \begin{equation*}
        \frac{\partial \rho}{\partial t} + \bm{\nabla} \cdot \bm{j} = 0
    \end{equation*}
    以及流体力学中的扩散系数
    \begin{equation*}
        \bm{j} = -c \bm{\nabla}\rho(\bm{x},t)
    \end{equation*}
    连续性方程可以等价为
    \begin{equation*}
        \rho(\bm{x_0},0)\mathrm{d}\bm{x_0} = \rho(\bm{x_t},t)\mathrm{d}\bm{x_t} 
    \end{equation*}

    现在希望从物质流密度的定义出发来推导连续性方程。由Gauss公式。
    \begin{align*}
        \int \bm{\nabla} \cdot \rho(\bm{x},t) \bm{\dot{x}}(\bm{x},t) \mathrm{d}\bm{x} = \int \rho(\bm{x},t) \bm{\dot{x}}(\bm{x},t) \cdot \mathrm{d}\bm{S}
    \end{align*}
    根据边界条件,
    \begin{align*}
        \int \rho(\bm{x},t) \bm{\dot{x}}(\bm{x},t) \cdot \mathrm{d}\bm{S} = \int -\frac {\partial \rho(\bm{x},t)}{\partial t} \mathrm{d}\bm{x}
    \end{align*}
    这要求物质只能通过表面来进出积分区域。由此可以推出
    \begin{align*}
        \int \bigg(\frac {\partial \rho}{\partial t} + \bm{\nabla \cdot j}\bigg) \mathrm{d}\bm{x} = 0 
    \end{align*}
    从对物质的密度的积分出发
    \begin{align*}
        \int \rho \mathrm{d}x = C
    \end{align*}
    对时间求全导数,得
    \begin{align*}
        \int \frac {\mathrm{d}}{\mathrm{d}t}\rho(\bm{x_t},t)\mathrm{d}\bm{x_t} = 0
    \end{align*}
    所以
    \begin{align*}
    \int \frac {\mathrm{d}}{\mathrm{d}t}\rho(\bm{x_t},t)\mathrm{d}\bm{x_t} &= \int \frac {\mathrm{d}\rho}{\mathrm{d}t}\mathrm{d}\bm{x_t} + \rho \mathrm{d}\bm{\dot{x}}
    \end{align*}
    考虑不同时刻的体积元
    \begin{equation*}
        \mathrm{d}\bm{x_t} = \mathrm{d}\bm{x_0} \bigg|\frac {\partial \bm{x_t}}{\partial \bm{x_0}}\bigg|
    \end{equation*}
    计算这个Jacobi行列式,就对Jacobi矩阵求导
    \begin{align*}
        \frac {\mathrm{d}}{\mathrm{d}t} \frac {\partial \bm{x_t}}{\partial \bm{x_0}} = \frac {\partial \bm{\dot{x}_t}}{\partial \bm{x_0}} = \frac {\partial \bm{\dot{x}_t}}{\partial \bm{x_t}} \frac {\partial \bm{x_t}}{\partial \bm{x_0}}
    \end{align*}
    又有初始状态
    \begin{align*}
        \bigg|\frac {\partial \bm{x_t}}{\partial \bm{x_0}} \bigg|\Bigg|_{t=0} = 1
    \end{align*}

\section{20210104:Bohmian动力学}

    类比经典情况有守恒的连续性方程
    \begin{equation*}
        \frac {\partial \rho}{\partial t} + \bm{\nabla} \cdot (\rho \bm{v}) = 0
    \end{equation*}
    在量子力学中也有守恒量:
    \begin{equation*}
        \int |\psi(\bm{x},t)|^2 \mathrm{d}x = 1
    \end{equation*}
    因此上述连续性方程依然成立。

    考虑含时Schrodinger方程
    \begin{equation*}
        \mathrm{i}\hbar \frac {\partial}{\partial t} \psi(\bm{x},t) = \bigg(\frac {\hbar^2}{2m}\bm{\nabla + \bm{v}}\bigg)\psi
    \end{equation*}
    波函数可以写成
    \begin{equation*}
        \psi(\bm{x},t) = \sqrt{\rho} \mathrm{e}^{\frac {\mathrm{i}S(\bm{x},t)t}{\hbar}}
    \end{equation*}
    将波函数的形式代入Schrodinger方程可以得到上式以及Hamilton-Jacobian方程
    \begin{align*}
        \frac {(\frac {\partial S}{\partial x})^2}{2m} + V(\bm{x}) - \frac {\hbar^2}{2m} \frac {\nabla^2 \sqrt{\rho}}{\sqrt{\rho}} = -\frac {\partial S}{\partial t}
    \end{align*}
    前两项可以和经典情况类比,第三项是量子力学而来,定义
    \begin{equation*}
        Q(\bm{x},t) = - \frac {\hbar^2}{2m} \frac {\nabla^2 \sqrt{\rho}}{\sqrt{\rho}}
    \end{equation*}
    为\textbf{量子势}。速度场即为
    \begin{equation*}
        \bm{v} = \bm{M}^{-1}\bm{p} = \bm{M}^{-1}\frac {\partial S}{\partial \bm{x}}
    \end{equation*}
    从Hamilton-Jacobian方程出发,可以得到运动方程。如果是经典的Hamilton-Jacobian方程,没有量子势
    \begin{equation*}
        \frac {(\frac {\partial S}{\partial x})^2}{2m} + V(\bm{x}) = -\frac {\partial S}{\partial t}
    \end{equation*}
    考虑作用量的全导
    \begin{equation*}
        \frac {\mathrm{d}S}{\mathrm{d}t} = \frac {\partial S}{\partial t} + \frac {\partial S}{\partial \bm{x}_t} \dot{\bm{x}_t} = \frac 12 \bigg(\frac {\partial S}{\partial \bm{x}_t}\bigg)^{\mathrm{T}} \bm{M}^{-1} \frac {\partial S}{\partial \bm{x}_t} - V(\bm{x})
    \end{equation*}
    之后可以对位置求偏导,最终得到
    \begin{align*}
        \dot{\bm{x}}_t &= \bm{M}^{-1} \frac {\partial S(\bm{x}_t,t)}{\partial \bm{x}_t}\\
        \dot{\bm{p}}_t &= -\frac {\partial V(\bm{x}_t)}{\partial \bm{x}_t} - \frac {\partial Q(\bm{x}_t)}{\partial \bm{x}_t}
    \end{align*}
    这称为\textbf{量子轨线方程}。如果对$\rho$求全导,
    \begin{equation*}
        \frac {\mathrm{d}\rho}{\mathrm{d}t} = \frac {\partial \rho}{\partial t} + \frac {\partial \rho}{\partial \bm{x}_t} \dot{\bm{x}}_t = -\rho \bm{\nabla} \cdot \dot{\bm{x}}_t
    \end{equation*}

    波函数的求解可以转化为用0时刻的求解
    \begin{equation*}
        \psi(\bm{x},t) = \langle \bm{x} | \mathrm{e}^{-\frac {\mathrm{i}\hat{H}t}{\hbar}} | \psi(\bm{x},0) \rangle = \sum_n \langle \bm{x}|\phi_n \rangle \langle \phi_n |\psi_0 \rangle \mathrm{e}^{-\frac {\mathrm{i}E_n t}{\hbar}}
    \end{equation*}

    例如,
    \begin{equation*}
        \psi(x,0) =  \bigg(\frac {2a}{\pi}\bigg)^{\frac 14} \mathrm{e}^{-\alpha (x - x_{\mathrm{eq}})^2 +\frac {\mathrm{i}{\hbar}} p_{\mathrm{eq}}(x - x_\mathrm{eq})}
    \end{equation*}
    则
    \begin{align*}
        S(x_0,0) &= p_\mathrm{eq} (x_0 - x_\mathrm{eq})\\
        p_0 &= \frac {\partial S}{\partial x_0} = p_\mathrm{eq}
    \end{align*}

    根据经典情况统计物理的结果
    \begin{equation*}
        \langle A \rangle =\frac 1Z \int \rho(x,p)A(x,p) \mathrm{d}x\mathrm{d}p
    \end{equation*}
    其中 
    \begin{align*}
        \rho &= \frac 1Z \mathrm{e}^{-\beta H(x,p)}\\
        Z &= \int \frac 1{2\pi \hbar} \mathrm{e}^{-\beta H(x,p)} \mathrm{d}x \mathrm{d}p
    \end{align*}
    现在要问,量子力学有没有同样的形式?

    定义
    \begin{equation*}
        \mathrm{Tr} (\hat{A}\hat{B}) = \frac 1{2\pi\hbar}\int A(x,p)B(x,p) \mathrm{d}x \mathrm{d}p
    \end{equation*}
    重要的是求出两个算符在量子力学下的形式。
    \begin{align*}
        A_\mathrm{w}(x,p) &= \langle x - \frac {\Delta}2 | \hat{A} | x + \frac {\Delta}2 \rangle \mathrm{e}^{\frac {\mathrm{i}\Delta p}{\hbar}}\\
        B_\mathrm{w}(x,p) &= \langle x - \frac {\Delta}2 | \hat{B} | x + \frac {\Delta}2 \rangle \mathrm{e}^{\frac {\mathrm{i}\Delta p}{\hbar}}
    \end{align*}
    可以验证,如果$\hbar \to 0$, 应有$A_\mathrm{w} \to A_\mathrm{cl}$, $B_\mathrm{w} \to B_\mathrm{cl}$. 

    可以表示出概率密度分布
    \begin{equation*}
        \hat{\rho} = |\psi \rangle \langle \psi |
    \end{equation*}
    可以在位置空间和动量空间给出概率密度分布。并可以验证
    \begin{align*}
        \int \rho_\mathrm{w}(x,p) \mathrm{d}x &= \langle p|\hat{\rho}|p \rangle\\
        \int \rho_\mathrm{w}(x,p) \mathrm{d}p &= \langle x|\hat{\rho}|x \rangle
    \end{align*}
    要求物理量期望值可以写成
    \begin{align*}
        \langle \psi | \hat{B} | \psi \rangle &=  \langle \psi |\sum_n | \phi_n \rangle \langle \phi_n | \hat{B} | \psi \rangle\\
        &= \sum_n \langle \phi_n | \hat{B} | \psi \rangle \langle \psi  | \phi_n \rangle\\
        &= \mathrm{Tr} (\hat{B}|\psi \rangle \langle \psi |)
    \end{align*}
    它等价为
    \begin{equation*}
        \langle \psi|\hat{B}|\psi \rangle = \frac 1{2\pi \hbar} \int \rho_\mathrm{w}(x,p) B_\mathrm{w}(x,p) \mathrm{d}x\mathrm{d}p
    \end{equation*}    

\end{document}